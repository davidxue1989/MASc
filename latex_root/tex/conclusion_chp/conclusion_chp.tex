\chapter{Conclusion}
In this thesis, we aimed to investigate a new modality of interaction for COACH, using a humanoid robot NAO, during the prompting of hand-washing steps to a child with ASD.  A Wizard of Oz (WoZ) study was conducted, and yielded promising results.  It was demonstrated that the humanoid robot NAO was able to contribute in assisting the child through hand-washing.  The child's compliance to the robot prompts was initially low, but with sessions of the parent training the child to follow the robot, the child's compliance increased.  The WoZ study results suggest promising prospects of utilizing the humanoid as a novel prompting modality to enhance COACH, and to fill the gap of in-vivo ATCs for teaching daily skills that can automatically deliver prompts and reinforcement schedules.  Future investigations in regards to improving the child's engagement further through more creative design of the robot's appearance and behavior dynamics were suggested.

In addition to the WoZ study, we further attempted to improve COACH by implementing a Visual Focus of Attention (VFOA) tracker.  A head pose tracking algorithm was successfully implemented using the Kinect2 camera and by modifying the open source software KinFu.  However, its tracking capability was not evaluated.  Furthermore, the head pose tracking algorithm was not adequate to analyze the video footage collected of the child with ASD participant during hand-washing trials due to fast moving of the child's head.  Because of this, we did not continue the implementation of the eye pose tracking algorithm, but only fleshed out its method.  Lastly, a simple object location calibration method was proposed and implemented.  In whole, we have shown feasibility of using the Kinect2 camera for VFOA estimation, and raised important areas of investigations for future development of the algorithm using this framework.
\chapter{Conclusion}
In this thesis, we aimed to investigate a new prompting agent, humanoid robot NAO, for COACH during the prompting of hand-washing steps to a child with ASD.  A Wizard of Oz (WoZ) study was conducted, and yielded promising results.  In addition, we further improved COACH by attempting to implement a Visual Focus of Attention (VFOA) tracker.  The thesis answers the hypotheses raised in the following way (the hypotheses are shown in bold):

\paragraph{The humanoid robot, NAO, is able to independently assist child with ASD through hand-washing, and child exhibits greater engagement level, higher prompt compliance rate, and better task completion when prompted by NAO than by parent.}
Through the WoZ study, we have seen that NAO was effective in facilitating task completion in both variety and quality of steps, approaching the level of effectiveness achieved by the parent, but not better.  Also, NAO had a low prompt compliance rate compared to the parent during the first phase when it was introduced, but through a training phase of joint prompting with the parent, NAO resulted in a higher prompt compliance rate than before, comparable to that of the parent's.  We attribute the improvements of the robot effectiveness to the training phase, though several confounding variables such as learning effects, fatigue, and robot control were discussed.  In whole, although we have not achieved totally independent assistance using NAO, we have shown that NAO has very good potential of achieving independent assistance, given a longer and more intense training phase.

\paragraph{Gestural, gaze, and verbal are the essential modes of interactions present in the hand-washing prompting scenario between child with ASD and the prompting agent NAO.}
The WoZ study revealed that verbal instructions and pointing gestures are essential for a prompting agent to assist our participant, who knows the execution of each hand-washing step, but needs reminders of which step to execute.  In cases the child did need demonstrations for a step, NAO's limited dexterity as well as the slower motion speed of the motion demonstrations made it less effective compared to that of the parent's.  In addition, when the child is not complying, the parent increased the severity of the prompts, which the robot should implement in the future.  In terms of gaze behaviors during interactions, our participant generally avoided looking at the prompting agent when he knew what to do, and tends to look at the parent more than NAO when seeking help.  The child's gaze behavior was a poor indication of engagement.  Lastly, detection and understanding of verbal feedbacks from the child may be useful in predicting the engagement of our participant.

\paragraph{Using 3DMM and ALR for estimating head pose and eye pose, and using the Kinect camera, a classification rate of more than 80\% is achieved for estimating child's VFOA on NAO, monitor screen, soap, towel, tap region, hands, and idling.}
We have successfully completed a head pose tracker using the Kinect camera by modifying the KinFu algorithm.  However, due to rapid head movements of the participant during hand-washing sessions, our head pose tracker was not successful in tracking his head.  Future improvements were suggested.  We have also implemented frontal pose transformation of the head image for eye region cropping.  But due to limitation of time, we did not implement the eye pose tracker using EYEDIAP dataset employing the ALR method.  Lastly, we implemented the object under gaze estimator.  In whole, we showed the feasibility of implementing the VFOA's three components (i.e. head pose tracker, eye pose tracker, object under gaze estimator), and described the steps for completing its implementation and evaluating its performance.

\section{Significance}
In conclusion, the thesis results suggest promising prospects of utilizing the humanoid robot, NAO, as a novel prompting agent to enhance COACH, and to fill the gap of in-vivo ATCs for teaching daily skills to children with ASD.  Future investigations in regards to improving the child's engagement further through more creative design of the robot's appearance and behavior dynamics were suggested and explored.  One such improvement could come from implementing robot behaviors contingent to the VFOA of the child.

\section{Looking Back}
Looking back, there are several things we could have done differently to have a more successful thesis.

Before the start of the WoZ study, if the researcher had a deeper and clearer understanding of qualitative analysis in case studies, the iterative qualitative data collection and analysis could have been carried more rigorously and efficiently, and possibly more yielded more results.  Starting the WoZ study earlier so we were less time limited would have been really nice, too.  That would possibly longer training phase, and possibly yielding better results, even demonstrating the robot being as effective as the parent in prompting.  Also, having a few child alone hand-washing trials would be ideal, since we could use that opportunity to assess the child's habits and skills before starting the robot prompting trials, and less trials would be wasted in the operator adjusting the robot prompting to the child's preference.  The researcher practiced controlling the robot to prompt adult volunteers through hand-washing during the study preparation phase, but the adults were very nice and all wanted to follow the robot.  Certainly improvements were made to the robot through this, but what we really needed was to test the robot for dealing with noncompliance behaviors, and testing the robot with normal children would have been much effective in regard.

During the study results analysis, we carried out quantitative analysis first, only relying on intuitions on what measures make sense to analyze.  After that, we carried out qualitative analysis and found better measures for quantitative analysis, and thus the quantitative analyses were revised in light of this.  Instead, if we did qualitative analysis first, our intuitions of what measures to be analyzed quantitatively could have been founded in the qualitative results to begin with, saving time from much revisions.

For the technical contribution section of the thesis, instead of implementing a head pose tracker, an eye pose tracker, and an object identification algorithm, we should decrease the scope to head tracking only.  This way, we could have more time to properly evaluate the accuracy and robustness of the algorithm implemented.
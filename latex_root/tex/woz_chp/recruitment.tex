\section{Recruitment}

Participants were recruited from a list of families of a previous autism study that indicated that they would be interested in participating in future studies related to the development of the COACH prompting system.

Participants were to be children between the ages of 4 to 15 with a diagnosis of ASD, and their parent. Three children would be recruited. This sample size is typical for studies of this nature for children with ASD, e.g. Kozima et al. had two participants in
\cite{kozima2005interactive}, Robins et al. had three in \cite{robins2004robot} and \cite{robins2009isolation}.  Participant demographics would be recorded and would include age, sex, and the Social Responsiveness Scale (SRS) test results.  The SRS is a commonly used tool to identify the presence and estimate the severity of ASD \cite{constantino2002social}. The results of the SRS would allow the research team to substantiate a diagnosis of an ASD for the child participants before proceeding with the study.

\paragraph{The \textbf{inclusion criteria} for enrolling in the study were as follows:}
\begin{itemize}
	\item Boys and girls between the ages of 4-15
	\item Parent report of a clinical diagnosis of an ASD – to be confirmed through administration of the Social Responsiveness Scale (SRS)
	\item Has difficulty independently completing self-care activities, specifically hand-washing
	\item Has the ability to follow simple, one-step verbal instructions
	\item Ethical consent  granted by parents or primary guardian
	\item Does not exhibit severely aggressive behavior
\end{itemize}

Each participating family were given a \$200 honorarium per child subject upon completion of the study. All participants were able to withdraw from the study at any time. The honorarium would then be adjusted to be proportionate to the number of visits completed (e.g. completing 3 visits means the participated child would receive \$100 (\$200 * 3 / 6 = \$100)). This would be made clear to participants at the time of consent.

\section{Data Analysis and Results}
In the following section, the hand-washing trials, the SRS survey, entrance survey, and post-interventions survey, and the parent interviews are analyzed qualitatively, and the annotated video data of the hand-washing trials is analyzed quantitatively.  

\subsection{Participants Recruited}
Due to limitation of time, we were only able to recruit one subject.  Our participant is a thirteen years old male child of Asian ethnicity.  He was accompanied to the trials by his mother, who was the one that answered all surveys.

\subsubsection{Entrance and SRS Survey Results}
The following backgrounds of the child were reported by the entrance and SRS surveys.  The purpose of this section is to familiarize the reader with our participant and understand better the context that our study results situate in.

\paragraph{Child's Demographics and Inclusion Criteria Fit}
The parent reported that the child has been clinically diagnosed of Autism Spectrum Disorder.  We also conducted the Social Responsiveness Scale Survey, and the child obtained a T-score of 79, passing the minimum score for severe ASD.  Through the Entrance Survey, we learned that the child has difficulty independently completing self-care activities (a 2 on a scale of 1 (not independent at all) to 5 (completely independent)), and this includes hand-washing (also a 2 on the same scale).  We also learned that the child is able to but not good at verbal communication (a 3 on a scale of 1 (very not well) to 5 (very well)).  Specifically, the child can only speak one or two words at a time to express what he wants, uses iPad for communication, and often just murmurs illegibly.  However, the child has the ability to follow simple, one-step verbal instructions (a 4 on a scale of 1 (very not well) to 5 (very well)).  Lastly, the child does not exhibit severely aggressive behavior (a 1 from a scale of 1 (never) to 5 (often)).  The above shows that the child fits our inclusion criteria.

\paragraph{Child's Experience with Technologies}
The childf is more of a visual learner.  He uses a computer at home, and likes to use it very much.  He also likes to use other technologies (e.g. iPhone, iPad), and likes to watch movies and TV.  He doesn't have a robot toy to play with at home or at school, so the parent doesn't know how much he likes to play with robot toys.  The parent never used technologies to help the child with self-care activities except using pictures to teach step by step hand-washing.

\paragraph{Child's Personal Preferences}
The child is sensitive to sound.  He likes Disney cartoon musics, and likes to watch his favorite cartoon scenes repeatedly on the iPad.  To reward the child after a good behavior, the parent suggested the following rewards: give extra time to play on iPad, give him praises (e.g. good job), give children books to read and animal dinosaurs to play with, give the parent's iPhone since the child's favorite musics are on there.

\paragraph{Child's Abilities on Hand-washing and on Other ADLs}
The parent agrees that the child usually gets distracted when performing hand-washing.  To assist him, the parent mainly reminds him to put soap, rinse properly, and dry properly with towel.  He needs more prompting in these areas since he always washes in a hurry.

Other activities the child needs help with include:  tooth brushing -- 2 hours/week; bathing -- 4.5 hours/week; dressing -- usually just hand the clothes to him, he knows how to put them on, but needs reminders of the order of the clothing, 7 hours/week.

\paragraph{Parent's Expectation and Concerns}
The parent expected the robot to be helpful in reminding the child to put soap, rinse and dry more, similar to the role of M.  Some concerns the parent has include: the child may be wondering why does he need to wash hands so many times repeatedly; the child performs well in his comfort zone with the same environment, so it takes a while for the child to get used to the lab environment.

\subsection{Case Selected}
Our participant fits the inclusion criteria as a user with typical use case for the robot.  It was noted that the typical use cases for the robot may be several, depending on the kinds of helps the user needs.  From the background report above, we understood that the child mainly needs help initiating the hand-washing activity, and guidances during certain steps (i.e. the get soap, rinse, and dry steps), but the child does not need much prompts for demonstrating the actions of each step.  This tendency of the child was verified during the trials, and we attempted to answer the question how ``best to prompt the child'' given this tendency (e.g. Should we prompt only the steps the child needs help, or prompt every step to create consistency?  Should we still demonstrate actions for the prompted steps or skip the MoDemo part of the prompt entirely?)  These will be answered in our discussions.

\subsection{Qualitative Data Analysis Method}
The sources of data collected for qualitative data analysis included child hand-washing trials videos, between trials parent researcher interview transcriptions, and after trial parent survey.  With these data, as well as researcher's self reflections, we would attempt to answer what are the parent's and researcher's thoughts on how did the child receive the robot as a prompting agent and how to improve the robot, the prompting protocol, and the research study.

The child hand-washing trials videos were only used as a reference to go back to if uncertainties arose in analysis that needed verifications.  They were not used as a primary source of data to generate themes.  Instead, the process of making field notes based on observations of the child's behaviors during the hand-washing trials were achieved by the parent researcher discussion interviews during the breaks between trials.  Then, the transcriptions of parent researcher interviews were analyzed from trial to trial, and the ``constant comparative method'' was applied to distill common categories or themes that cut across all trials.  In concise words, the ``constant comparative method'', as described by Merriam, is an analysis method commonly used in qualitative researches that involves comparing ``one unit of information with the next in looking for recurring regularities in the data'' \cite{merriam2014qualitative}.  Categories are formed that the data relevant to the research questions fit into.  And as we go through more units of information, these categories may be subdivided or subsumed under more abstract categories as we ``begin to discriminate more clearly between the criteria for allocating data to one category or another''.  As a side note, this process focuses on describing the data in a concise and logical way rather than on evaluating hypotheses.

Lastly, themes distilled from the parent's post intervention survey were reported on improving robot, prompting protocol, and research study.  In addition, the researcher would self reflect on these topics and give inputs as the person who operated the robot and conducted the study.

%this is the qualitative data results section
\subsection{Qualitative Description and Analysis Results}
In this section, qualitative data analysis results are reported.  We first compare and contrast between parent prompting and robot prompting on how the prompts were delivered and how well did the child receive them.  Then we examine the factors contributing to the child's difference of reception of robot prompts and the parent prompts.  Lastly, we explore changes to the robot and the study that possibly improve robot prompt reception from the child.

\subsubsection{Parent Prompts - Phase A}
During Phase A, where the parent prompted the child through hand-washing alone, the parent was not instructed on what to prompt or how to prompt by the researcher.  The only input given to the parent during this phase was to get the child to rinse more and dry more.  The parent had no prior knowledge of how the robot prompts are like either.  Thus, the way the parent prompted was very similar to how she'd have prompted at home in everyday settings, and one she probably found effective to her son.

The sequence of hand-washing steps the parent prompted was in the following order: get soap, lather, turn on water, rinse, turn off water, dry.  This order is one of the many logical orders of hand washing, since one could alternatively choose to turn on water before lather so one could lather in the running water, or even to turn on water before getting soap, so one's hand is wet when lathering.  However, the parent chose to get soap and lather first before turning on water, probably because of her habit, and had not changed this sequence order through out the trials in Phase A.  Although, an interesting thing to note is, whenever the parent did not prompt for lather right after get soap and leave the child to decide which steps to do, the child would always turn on the water first and lather in water.

The parent used verbal prompts as the major prompting modality, although for some steps, she also prompted using gestures such as pointing and motion demonstration.  The parent's prompting patterns most common for each step are summarized in Table \ref{tab:ParentPrompts}.  In the table, ``MoDemo'' refers to motion demonstration gestures (e.g. rubbing hands together for lather step and flipping and turning hands in the air for dry step).  ``Point'' refers to pointing to the object of interaction (e.g. the water during rinse step).

\begin{table}[H]
	\centering
	\begin{tabular}{ | l | l | l | }
		\hline
		\textbf{Step}	&	\textbf{Verbal Prompt} & \textbf{Gesture Prompt}	\\	\hline	\hline
		
		Get soap	&	Get some soap; Soap;	&	- 	\\	\hline
		Stop soap	&	Just one; That's it; Enough; That's too much;	&	-	\\ \hline
		Lather	&	Lather your hands; Lather; Soap;	& MoDemo	\\	\hline
		Turn on water	&	Turn on the water;	&	-	\\	\hline
		Rinse	&	Rinse more; Water; More water; Wash hands; More;	&	MoDemo; Point	\\	\hline
		Turn off water	&	Turn off the water;	&	-	\\	\hline
		Dry	&	Dry; More Dry; Towel; Dry your hands; More;	&	MoDemo	\\	\hline
		
	\end{tabular}
	\caption{Parent Prompts}
	\label{tab:ParentPrompts}
\end{table}

For the lather, rinse, and dry steps, they are not single action steps, but need continuous motion.  For these steps, the parent would repeatedly deliver verbal prompts of the same step every two seconds or so, until the parent was satisfied with the step's duration.  Variations of the step's verbal prompt were often used (e.g. shortening ``dry your hands'' to ``dry'', or add the word ``more'' or simply using ``more'' for verbal prompt).  The variations can be seen in Table \ref{tab:ParentPrompts}.  Most often for the lather step, and sometimes for the rinse and dry steps, the parent also delivered continuous motion demonstration, and would verbally prompt ``like this'' for the child to follow.  Note that the parent used ``soap'' for verbal prompt of both get soap step and lather step, but the child understood what to do since the parent always delivered motion demonstrations for the lather step.

The timing of prompt deliveries can be categorized as the following:
\begin{itemize}
	\item \textbf{Proactive Prompt}: prompt a new step before the child finishes current step
	\item \textbf{Corrective Prompt}: prompt the same step when the child is doing or is about to do a wrong step
	\item \textbf{Reactive Prompt}: prompt when the child is waiting for prompts
	\item \textbf{No Prompt}: letting the child do what he wants without objection
\end{itemize}
The parent mostly delivered proactive prompts, prompting the next step right when or before the child finished the current step.  However, sometimes corrective prompts were needed when the child was not complying or executing correctly.  There were also times that reactive prompts were needed, for example during rinsing, when the child terminated the step early but was waiting for the parent's permission to go on to the next step.  The parent either told the child to keep doing the current step or acknowledged that the child can proceed to the next step.  The last and the least common scenario was the parent delivering no prompts.  This happened when the parent waited to see if the child knew what to do without being prompted, and the child either executed the correct step for the appropriate duration, needing no prompts, or executed incorrectly or for too short or too long of a duration, but the parent let it go without correction.

In addition to the step prompts, the parent also delivered verbal rewards, but usually only at the very end of each hand-washing trial when the child is all done, saying ``good job''.  Also, during the hand-washing, the parent's gaze was mainly focusing on the object being prompted, and very little eye contacts between the parent and the child were seen.  However, during the verbal reward after the child is all done, the parent would often initiate eye contacts.

The child was generally good in complying to the parent's prompts, although at times noncompliance still occurred.  Systematically, compliance can be categorized into the following four scenarios:
\begin{enumerate}
	\item Waiting to be prompted before executing a step
	\item Executing a step as prompted
	\item Keep executing a step until prompted otherwise or verbally awarded
	\item Terminating a step after being prompted for another step
\end{enumerate}
And conversely, noncompliance are scenarios outside of the compliance scenarios.  There are five common noncompliance scenarios:
\begin{enumerate}
	\item Executing a step before prompted
	\item Executing the wrong step after prompted
	\item Idling after prompted
	\item Terminating a correct step before prompted for another step or verbally awarded
	\item Not terminating a correct step after prompted for another step
\end{enumerate}

Specifically, the child was able to wash hands well under the parent's prompting.  The child would usually start the get soap step first, and he'd start the step by himself if the parent did not proactively prompt it.  There were times, however, when the child pressed the soap sprout too many times, and the parent had to prompt him multiple times for him to stop.  The lather step was prompted proactively by the parent with MoDemo gestures, and the child followed it obediently for the whole duration, without terminating the step by himself.  The turn on water step is easy for the child, he'd do it by himself right after the get soap step and lather in water if the parent didn't proactively prompt, or he'd execute the turn on water step after lathering if the parent prompted so.  The child would always immediately start rinsing by himself after turning on water, without waiting for the rinse prompt.  He could execute the motion of this step well, but the challenge was for him to rinse longer.  After about eight seconds of rinsing in water, he'd terminate the step by himself, and wait for the parent to prompt the next step.  If the parent continued to prompt for the same step (i.e. rinse more), he'd put his hands in the water for one second and withdrew it and wait for the next prompt.  However, he would not start the turn off water step by himself without getting prompted or approval from the parent.  Once the parent prompted for turn off the water, the child had no problem following the prompt and executing it.  In addition, he would always start the drying step by himself immediately after turning off the water.  The drying step was a little challenging for the child in both its motion and duration.  The child's motion was mostly drying the inside of his hands, without turning the towel or hand to dry the outside.  However, the parent only demonstrated the motions sometimes, and the child did not change his actions because of the MoDemo gesture prompts.  The duration of the drying step was around eight seconds before the child decided to terminate the step by himself.  The parent only proactively prompted dry more, but did not object when the child decided to terminate the step by putting down the towel.  The parent instead nodded and approved the child to leave the sink once he put down the towel.

The one step the child was openly not complying to the parent's prompt is the get soap step, where the parent told the child to stop many times, but the child still kept pumping the soap onto his hands.  Also, sometimes the child did not want to rinse for the full duration despite the parent prompting rinse more.  The strategies the parent used in dealing with these noncompliance behaviors were:
\begin{itemize}
	\item Proactively prompting before the child executes wrongly
	\item Repeatedly prompting with no pause
	\item Mention the child's name in the verbal prompt
	\item Shortening the verbal prompt into a one or two word phrase and increase in severity of tone when repeating the prompt
	\item Demonstrating the motion with fast moving and continuous gesture prompt
	\item Physically intervene by guiding the child's hands in executing the correct actions
\end{itemize}
The first five strategies were mostly working, and only rarely did the parent need to physically intervene (only for stopping the get soap step).

In addition, sometimes the child played with the soap bubbles in the sink during hand-washing.  The parent prompted ``no playing with the bubbles'', and the child would usually comply.  However, there was once when the child played with the soap bubbles after rinsing, so the parent prompted for him to rinse again.  However, the child was confused in or did not want to comply to what the parent wanted, and started washing hands from the get soap step once more.  If the child was not confused but was simply noncompliant, it may suggest the child could not wash hands from the middle, but would always need to wash hands in a linear fashion from the start.

The child seemed to experience fatigue after few trials into each visit, although we had breaks between each trial for around three minutes each.  The fatigue behaviors inlcude playing with bubbles, getting too much soap, terminating rinse steps early, and rushing through the dry steps, and were found more likely by the later trials of each visit.

Lastly, the child had a tendency to repeat the verbal prompts after hearing it.  The words he said were not very clear, it's almost like a murmur, but his parent confirmed he was repeating the verbal prompt so he can process it.

\subsubsection{Robot Prompts - First Phase B}
The robot prompting trials spanned three phases, namely and in order: the first Phase B, Phase C, and the second Phase B.  The robot prompts were described in Section \ref{sec:SpecificProtocol}.  However, improvisations from the operator as well as changes to the robot behavior and study protocol were needed as the study progressed.

In the first Phase B, the robot was introduced to the child for the first time, and it prompted alone in the washroom.  The operator tried his best to prompt for all seven steps (i.e. turn on water, wet hands, get soap, scrub, rinse, turn off water, dry).  However, due to unfamiliarity with and the lack of ease of the robot control interface and unfamiliarity with the child's behaviors, the operator could not control the robot to keep up with the child's pace.  There were often four to eight seconds of pauses between each new robot prompts, and one to two seconds of pauses between repeats of the same prompt.  In addition, due to the way the robot gestures were implemented, each robot prompt was around four seconds to execute.  Thus, the child could have executed several steps on his own before the robot prompted another step.  And this was often the case, requiring the operator to plan and predict the child's behavior ahead of time in order to make the robot prompts relevant and up to the child's current progress in the hand-washing activity.  It also required the operator to refrain from prompting all seven steps, but to only choose three or four important steps to prompt.

The operator was able to focus on delivering the prompts for turn on water, get soap, and rinse steps.  The operator experimented with prompting turn on water as first step against prompting get soap as first step.  However, the child was determined to get soap first, despite the robot's repeated prompts of turn on water.  In one trial, the parent intervened verbally, but it was not until three repeated verbal prompts from the parent that the child complied.  And the compliance did not persist, the trial after that, the child immediately reverted back to getting soap first regardless of robot prompts.  One thing to note, however, is for the trials that the robot did prompt get soap as first step, the child would not respond immediately or would start the step before waiting to be prompted.  The operator tried to prompt turn on water and get soap steps proactively.  However, there was a introduction prompt (i.e. ``Hi [child's name], let's start washing hands'') before the first step, and the pause between this prompt and the first step prompt was sometimes long (around two seconds), and the child decided to start by himself.  Next, the operator was not able to prompt the wet hands or scrub steps because the child would start rinse step immediately following turn on water step by himself.  For rinse step, due to robot delay, there was usually a four to eight seconds pause between the previous prompt and the rinse prompt (the longer eight seconds pause happened when the robot gave verbal reward for the previous step, which took three seconds or so).  The child had finished rinsing during this pause (the child usually rinsed four seconds before stopping in these first Phase B trials), and would just wait for the robot's prompt.  However, the intriguing thing was, when the robot started prompting the rinse step, the child would take it as a cue to turn off water and start the dry hands step.  This had been the case for all trials, and the child's behavior would not change even when the robot repeatedly prompted the rinse step or the parent came in to verbally intervene.  The child may be misunderstanding the robot's rinse prompt, or may be simply disobeying its prompt.  But as a consequence, the child finished hand washing before the robot could prompt any more steps after rinse, and thus the operator did not have a chance to prompt turn off water or dry hands steps.

The first step was prompted proactively, but most of the other steps were prompted reactively rather than proactively due to the huge delays in the robot prompt deliveries.  There were some corrective and no prompts, but not often.  The child's compliance to the robot prompts were minimal.  Though the child appeared to be waiting for the robot prompts in the very first couple of trials, the child quickly lost interest in doing so because of the lack of responsiveness of the robot.  Instead, the child mostly ignored the robot washed hands on his own for the trials after, except for the rinse step, where the child waited for the robot after terminating the step, and moved on to next step after the robot prompted (although the robot prompted to rinse hands more instead of moving on).  The child's noncompliance behaviors were found in all five scenarios listed previously in the parent prompt section.

The robot's strategies to deal with noncompliance from the child included repeating the prompt and the use of attention grabber (AG) by calling the child's name.  AG did not work, the child never looked over at the robot because of AG.  Repeating the prompt did not work either, and noncompliance continued even with the parent coming in and verbally prompt as well.

The child's gaze was on the robot more in the first couple of trials, but in later trials the child did not look at the robot much.  This gaze behavior of the child seemed to correlate with whether he was waiting for the robot.  In light of this, it might be the case that the gaze behavior of the child on a prompting agent is an indication of the child's level of interest, which correlates to the child's willingness to wait for the agent's prompts.

Verbal rewards were used more often in the first few trials.  However, because it delayed the delivery of the next prompts, the operator decided to decrease the frequency of verbally rewarding the child in the later trials.  This did not seem to have an effect on the child's behaviors, it only made the robot prompt deliveries quicker.

\subsubsection{Robot Prompts - Phase C}


\subsubsection{Robot Prompts - Second Phase B}

\subsubsection{Factors for Child Non-Compliance}

\subsubsection{Possible Improvements for Compliance}


\subsubsection{Post-Intervention Survey for Parent}
The post-intervention survey results are presented in Table \ref{tab:PostInterventionSurveyData}.  We see that the parent was happy with all aspects of the robot as a prompting agent, but did not see it to be as good as herself yet.  The suggestions the parent made regarding the robot and the experiment are reported in the discussion section.

\begin{table}[H]
	\centering
	\begin{tabular}{ | p{12cm} | l | }
		\hline
		\textbf{Survey Question}	&	\textbf{Parent's Answer}	\\	\hline	\hline		
		Hand-washing steps break down was appropriate	&	Strongly Agree	\\	\hline
		My child understood the verbal prompts	&	Agree	\\	\hline
		Robot's verbal prompts were appropriate	&	Strongly Agree	\\	\hline
		The prompt wordings were similar to mine	&	Strongly Agree	\\	\hline
		The prompt voice and tone were appropriate	&	Strongly Agree	\\	\hline
		The prompt wordings were easy to understand	&	Strongly Agree \\	\hline
		My child understood the gesture prompts	&	Strongly Agree	\\	\hline
		The gesture prompts were appropriate	&	Strongly Agree	\\	\hline
		The gesture prompts were easy to understand	&	Agree	\\	\hline
		The physical appearance of robot is aesthetically pleasing	&	Agree	\\	\hline
		The attention grabber gestures were appropriate	&	Strongly Agree	\\	\hline
		The verbal rewards were appropriate	&	Strongly Agree	\\	\hline
		The reward gestures were appropriate	&	Strongly Agree	\\	\hline
		The robot was effective in assisting my child through hand-washing	&	Strongly Agree	\\	\hline
		The robot motivated my child to wash hands	&	Strongly Agree	\\	\hline
		The robot was fun for my child to use	&	Strongly Agree	\\	\hline
		My child was confused by the robot	&	Disagree	\\	\hline
		I like the idea of a robot prompting my child	&	Strongly Agree	\\	\hline
		The robot is able to provide guidance as well as I can or better	&	Neither Agree or Disagree	\\	\hline
		I would want to own a robot like this one	&	Strongly Agree	\\	\hline
	\end{tabular}
	\caption{The Post-Intervention Survey Data}
	\label{tab:PostInterventionSurveyData}
\end{table}


%rename this section to simply report sample selection
%when should i discuss parent's decision on how to involve during phase C?  should have been in the qualitative results section
\subsection{Experiment Design Change}
We were not able to counterbalance the confounding effect of learning through randomly assigning participants to phase orders (A-B-C versus A-C-B), since we only recruited one participant.  Instead, we decided to control it by splitting the Phase B into two segments, one before Phase the child and one after.  Thus, we conducted the study in the following phase order: A-B-C-B (i.e. parent alone phase - robot alone phase - robot parent phase - 2nd robot alone phase).  This way, we can compare first Phase B and second Phase B to see how much does learning in Phase the child affect our results.  Also, first Phase B and second Phase B won't have sixteen trials each due to limitation of time.  Instead, we conducted these two phases only long enough to see a stable response, as is typically done in a single-subject research design \cite{ayres2009acquisition, bereznak2012video}.  As a result, we had 16 trials for Phase A, 8 trials for first Phase B, 21 trials for Phase C, and 5 trials for second Phase B.  Note that the intervention conditions of first Phase B and second Phase B are meant to be the same.


\subsection{Video Data Analysis and Results}
\label{sec:VideoDataAnalysisAndResults}

\subsubsection{Analysis Method}
The analyses employed here are visual analyses.  The analyses mainly compare the levels (eyeballing the mean) of measures in different phases.  In cases where the level of a measure changes dramatically within a phase, this trend with initial level and final level were noted.

\subsubsection{Prompt Effectiveness}
To reflect how effective our prompting system is, we show whether the system can reduce both the number of incomplete steps and the number of parent prompts.

\paragraph{Number of Incomplete Steps}
We assumed that parent prompts had a higher level of authority over the child than robot prompts, because the robot only delivered verbal prompts and gestures such as pointing and motion demonstrations, while the parent could deliver those as well as nudging, guiding the arm, and completely executing the step for the child if the verbal and gesture prompts did not work.  This means we can measure the effect of robot prompts by comparing number of complete steps without prompts vs. with robot prompts alone, and measure the effect of parent prompts by comparing number of complete steps with robot prompts alone vs. with robot and parent prompts.  Figure \ref{fig:TotalNumberOfCompleteSteps} shows a series of plots for the measure ``Total Number of Complete Steps'', differing in what steps counted as completed when plotting the figures, e.g. Plot \ref{fig:7TotalNumberofCompleteSteps-WithoutPrompts} (``Without Prompts'') counts only steps completed by child with no prompts from the robot or the parent.  The next Plot \ref{fig:6TotalNumberofCompleteSteps-WithRobotPrompts} (``With Robot Prompts Only''), allows steps prompted by the robot to also count towards completed steps, and Plot \ref{fig:4TotalNumberofCompleteSteps-WithRobotAndParentPrompts} (``With Robot and Parent Prompts'') counts every completed steps even if they were prompted by the parent or the robot.

The most important comparison is between Plot \ref{fig:7TotalNumberofCompleteSteps-WithoutPrompts} (``Without Prompts'') and Plot \ref{fig:6TotalNumberofCompleteSteps-WithRobotPrompts} (``With Robot Prompts Only''), demonstrating the effectiveness of the robot's presence.  We see that parent alone phase (Phase A) was unaffected since robot wasn't present, but introducing the robot in the rest of the phases show effectiveness: robot alone phase (first Phase B) from 2.5 to 3, robot parent phase (Phase C) from 2 to 3, and robot alone repeat phase (second Phase B) from 2.5 to 4.

Comparing Plot \ref{fig:6TotalNumberofCompleteSteps-WithRobotPrompts} (``With Robot Prompts Only'') against Plot \ref{fig:4TotalNumberofCompleteSteps-WithRobotAndParentPrompts} (``With Robot and Parent Prompts''), we see the effectiveness of parent prompts: parent alone phase moved from 3 to 6, robot alone phase from 3 to 3.5, robot parent phase from 3 to 5, robot alone repeat phase from 4 to 4.5.
\begin{figure}[h]
	\centering
	\begin{subfigure}[b]{0.49\textwidth}
		\includegraphics[width=1.1\linewidth]{./img/data_analysis/110NumberofCompleteSteps-WithoutPorR.eps}
		\caption{Total Number of Complete Steps - Without Prompts}
		\label{fig:7TotalNumberofCompleteSteps-WithoutPrompts}
	\end{subfigure}
	\hfill
	%	~ %add desired spacing between images, e. g. ~, \quad, \qquad, \hfill, or double enter etc.	
	\begin{subfigure}[b]{0.49\textwidth}
		\includegraphics[width=1.1\linewidth]{./img/data_analysis/109NumberofCompleteSteps-WithR.eps}
		\caption{Total Number of Complete Steps - With Robot Prompts Only}
		\label{fig:6TotalNumberofCompleteSteps-WithRobotPrompts}
	\end{subfigure}%
	
	
	\begin{subfigure}[b]{0.49\textwidth}
		\includegraphics[width=1.1\linewidth]{./img/data_analysis/blank.png}
	\end{subfigure}%
	\hfill
	\begin{subfigure}[b]{0.49\textwidth}
		\includegraphics[width=1.1\linewidth]{./img/data_analysis/108NumberofCompleteSteps-WithPandR.eps}
		\caption{Total Number of Complete Steps - With Robot and Parent Prompts}
		\label{fig:4TotalNumberofCompleteSteps-WithRobotAndParentPrompts}
	\end{subfigure}%
	\caption{Total Number of Complete Steps}
	\label{fig:TotalNumberOfIncompleteSteps}
\end{figure}

\paragraph{Number of Parent Prompts}
The measure ``Total Number of Parent Prompts'' is plotted in Figure \ref{fig:TotalNumberOfParentPrompts}.  Plot \ref{fig:25TotalNumberofParentPrompts} is for the overall count (i.e. counting both physical and non-physical prompts).  It shows that during parent alone phase (A), the measure has an upward trend from 5 moving to 20.  However, in robot alone phase (first Phase B), we have a sudden drop leveling at near zero.  In robot parent phase (C), the measure has a downward trend moving from 15 to 5.  In robot alone repeat phase (second Phase B), we again observe a near zero level.  By comparing the measures across phases, we see that the robot's presence were effective in reducing the number of parent prompts, especially in robot alone and repeat phases.  Plot \ref{fig:26TotalNumberofParentPrompts-Physical} is for the physical prompt count.  This plot shows when the parent resorts to a higher prompt level (i.e. physical prompts such as nudging, guiding, and physically intervene) in order to get the child's compliance.  We see that the level is mainly near zero for all phases except for robot parent phase (C), leveling around 2.5.
\begin{figure}[h]
	\centering
	\begin{subfigure}[b]{0.49\textwidth}
		\includegraphics[width=1.1\linewidth]{./img/data_analysis/25TotalNumberofParentPrompts.eps}
		\caption{Total Number of Parent Prompts - Overall}
		\label{fig:25TotalNumberofParentPrompts}
	\end{subfigure}
	\hfill
	\begin{subfigure}[b]{0.49\textwidth}
		\includegraphics[width=1.1\linewidth]{./img/data_analysis/26TotalNumberofParentPrompts-Physical.eps}
		\caption{Total Number of Parent Prompts - Physical}
		\label{fig:26TotalNumberofParentPrompts-Physical}
	\end{subfigure}%
	\caption{Total Number of Parent Prompts}
	\label{fig:TotalNumberOfParentPrompts}
\end{figure}


\subsubsection{Child's Response to Prompts}
To illustrate the child's different responses to the prompts, we characterized child's responses into three categories: ``compliance'', ``not affected by prompt'', and others.

\paragraph{Compliance Rate}
A response is counted towards ``compliance'' if the child executes the correct step in response to the prompt.  If the child was executing the wrong step before prompt, and is converted into doing the correct step due to prompt, we call this hard compliance.  The compliance and hard compliance response rates are shown in Figure \ref{fig:ComplianceRate}.

Plot \ref{fig:102ComplianceRate-Overall} shows the overall compliance rate, with parent alone phase (A) leveling at 80\%, robot alone phase (first Phase B) leveling at 30\%, parent robot phase (C) moving upward from 60\% to 80\%, and robot alone repeat phase (second Phase B) leveling at 80\%.  We see that when the robot was first introduced in robot alone phase (first Phase B), the child did not comply to the prompts.  However, by going through Phase the child where the parent prompts for child to follow the robot, the child complies more readily in the robot alone repeat phase, achieving similar level of compliance as the parent alone phase.  We need to note that this plot includes prompts delivered by the robot, by the parent, and by them together.  Even in robot alone and repeat phases, the parent still comes into the washroom and prompts when the child isn't complying to the robot.  To see whether the robot alone can potentially guide the child through the whole hand-washing activity with minimal parent involvement, we plotted the compliance rate counted over only prompts delivered by the robot, shown in Plot \ref{fig:79ComplianceRate-R1Pv0g0}.  This plot confirms the levels observed in the overall plot, validating the improvement of compliance rate seen in R Alone Rep phase.  To investigate to what extent the child is compliant, the overall hard compliance rate is shown in Plot \ref{fig:103HardComplianceRate-Overall}, with parent alone phase (A) split leveling at 100\% and 35\%, robot alone phase (first Phase B) leveling at 25\%.  The robot parent phase (C) averaging around 60\% but the spread increases as trials went on.  Lastly, the robot alone repeat phase (second Phase B) levels at 50\%.  Similar to above, we observe an improvement of hard compliance between robot alone and repeat phases.  Looking at the robot prompts only Plot \ref{fig:92HardComplianceRate-R1Pv0g0}, the robot alone phase (first Phase B) drops to almost 0\%, while robot alone repeat phase (second Phase B) remains at 50\%.
\begin{figure}[h]
	\centering
	\begin{subfigure}[b]{0.49\textwidth}
		\includegraphics[width=1.1\linewidth]{./img/data_analysis/102ComplianceRate-Overall.eps}
		\caption{Compliance Rate - Overall}
		\label{fig:102ComplianceRate-Overall}
	\end{subfigure}
	\hfill
	\begin{subfigure}[b]{0.49\textwidth}
		\includegraphics[width=1.1\linewidth]{./img/data_analysis/79ComplianceRate-R1Pv0g0.eps}
		\caption{Compliance Rate - Robot Only Prompts}
		\label{fig:79ComplianceRate-R1Pv0g0}
	\end{subfigure}%
	
	
	\begin{subfigure}[b]{0.49\textwidth}
		\includegraphics[width=1.1\linewidth]{./img/data_analysis/103HardComplianceRate-Overall.eps}
		\caption{Hard Compliance Rate - Overall}
		\label{fig:103HardComplianceRate-Overall}
	\end{subfigure}
	\hfill
	\begin{subfigure}[b]{0.49\textwidth}
		\includegraphics[width=1.1\linewidth]{./img/data_analysis/92HardComplianceRate-R1Pv0g0.eps}
		\caption{Hard Compliance Rate - Robot Only Prompts}
		\label{fig:92HardComplianceRate-R1Pv0g0}
	\end{subfigure}%
	\caption{Compliance Rate}
	\label{fig:ComplianceRate}
\end{figure}

\paragraph{Not Affected By Prompt Rate}
A response is counted towards ``not affected by prompt'' if the child was executing a wrong step and did not change after the prompt or was idling and did not change after the prompt.  The not affected by prompt rate is shown in Plot \ \ref{fig:99NotAffectedByPromptRate-Overall}.  We see that for most phases, it levels at 15\%, but for robot alone repeat phase (second Phase B) it is at 35\%.  This shows the robot prompts were ignored more when the robot was first introduced, but improves to an acceptable level through Phase C.
\begin{figure} [h]
	\centering
	\includegraphics[width=0.6\textwidth]{./img/data_analysis/99NotAffectedByPromptRate-Overall.eps}
	\caption{Not Affected By Prompt Rate}
	\label{fig:99NotAffectedByPromptRate-Overall}
\end{figure}


\subsubsection{Engagement and Visual Attention}
To further characterize child's response to different prompting phases, we investigate how many times the child smiles and murmurs during step execution, and how often the child looks at the prompting agent during prompting and step execution.

\paragraph{Number of Times Child Smiles}
The measure ``Total Number of Times Child Smiles'' is shown in Plot \ \ref{fig:12TotalNumberofTimesChildSmiles}.  In it, parent alone phase (A) levels at 1.5, robot alone phase (first Phase B) levels at 0.5, robot parent phase (C) has a large spread and averages around 3, and robot alone repeat phase (second Phase B) also has a large spread and averages around 4.  It shows that the child smiles much more in later phases compared to earlier phases, and particularly, smiles in the repeat phase more than in robot alone phase.
\begin{figure} [h]
	\centering
	\includegraphics[width=0.6\textwidth]{./img/data_analysis/12TotalNumberofTimesChildSmiles.eps}
	\caption{Total Number of Times Child Smiles}
	\label{fig:12TotalNumberofTimesChildSmiles}
\end{figure}



\paragraph{Number of Times Child Murmurs}
The measure ``Total Number of Times Child Murmurs'' is shown in Plot \ \ref{fig:13TotalNumberofTimesChildMurmurs}.  In it, parent alone phase (A) has a large spread, averaging around 4.  Robot alone phase (first Phase B) levels at 0.5.  Robot parent phase (C) has a large spread, averaging around 4.  Robot alone repeat phase (second Phase B) levels at 2.  It shows that the child murmurs much more often when the parent is present.  Also, child murmurs in the repeat phase more than the robot alone phase.
\begin{figure} [h]
	\centering
	\includegraphics[width=0.6\textwidth]{./img/data_analysis/13TotalNumberofTimesChildMurmurs.eps}
	\caption{Total Number of Times Child Murmurs}
	\label{fig:13TotalNumberofTimesChildMurmurs}
\end{figure}

\paragraph{Looking at Prompting Agent Rate}
A prompt can be given by the parent, by the robot, or by them together.  During prompting and during step execution, the child may turn and look at the parent and/or the robot.  The gaze behavior of the child is shown in Figure \ref{fig:LookingAtPromptingAgentDuringPrompts} for all the cases above.  Because not all cases have the same amount of data in all phases, we will only mention here the phases that have enough evidence.  In Plot \ref{fig:8LookingatParentRateGivenParentPrompted}, ``Looking at Parent Rate - Given Parent Prompted'' levels at 40\% in parent alone phase (A).  In Plot \ref{fig:92HardComplianceRate-R1Pv0g0}, ``Looking at Robot Rate - Given Robot Prompted'' trends downward from 50\% to 30\% for robot alone phase (first Phase B), levels at 30\% with high spread for robot parent phase (C), and levels at 20\% for robot alone repeat phase (second Phase B).  We see that when the parent and the robot prompt individually, the parent has a higher chance of getting the child's visual attention.  Although the robot had similar levels of attention when was first introduced, it dropped as the study went on.  In Plot \ref{fig:10LookingatParentRateGivenBothPrompted}, ``Looking at Parent Rate - Given Both Prompted'' averages around 50\% with high spread in robot parent phase (C).  In Plot \ref{fig:11LookingatRobotRateGivenBothPrompted}, ``Looking at Robot Rate - Given Both Prompted' levels at 15\% for robot parent phase (C).  We see that when the parent and the robot prompt at the same time, the parent had a greater amount of visual attention.  It is interesting to note that the parent had similar levels of visual attention when prompting alone and when prompting with the robot.  The robot, however, experienced a decrease in visual attention level when the parent prompts with it.
\begin{figure}[h]
	\centering
	\begin{subfigure}[b]{0.49\textwidth}
		\includegraphics[width=1.1\linewidth]{./img/data_analysis/8LookingatParentRateGivenParentPrompted.eps}
		\caption{Looking at Parent Rate - Given Parent Prompted}
		\label{fig:8LookingatParentRateGivenParentPrompted}
	\end{subfigure}
	\hfill
	\begin{subfigure}[b]{0.49\textwidth}
		\includegraphics[width=1.1\linewidth]{./img/data_analysis/9LookingatRobotRateGivenRobotPrompted.eps}
		\caption{Looking at Robot Rate - Given Robot Prompted}
		\label{fig:9LookingatRobotRateGivenRobotPrompted}
	\end{subfigure}%
	
	
	\begin{subfigure}[b]{0.49\textwidth}
		\includegraphics[width=1.1\linewidth]{./img/data_analysis/10LookingatParentRateGivenBothPrompted.eps}
		\caption{Looking at Parent Rate - Given Both Prompted}
		\label{fig:10LookingatParentRateGivenBothPrompted}
	\end{subfigure}
	\hfill
	\begin{subfigure}[b]{0.49\textwidth}
		\includegraphics[width=1.1\linewidth]{./img/data_analysis/11LookingatRobotRateGivenBothPrompted.eps}
		\caption{Looking at Robot Rate - Given Both Prompted}
		\label{fig:11LookingatRobotRateGivenBothPrompted}
	\end{subfigure}%
	\caption{Looking at Prompting Agent Rate}
	\label{fig:LookingAtPromptingAgentDuringPrompts}
\end{figure}

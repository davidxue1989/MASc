\section{Case Study Format}

The WoZ study will be conducted, analyzed, and reported in a case study format.  A case study is defined by Yin in a two fold definition: "a case study is an empirical inquiry that investigates a contemporary phenomenon in depth and within its real-life context" and its result "relies on multiple sources of evidence, with data needing to converge in a triangulating fashion" \cite{yin2013case}.  The case study format is widely used in human robot interaction researches for children with ASD \cite{kozima2005interactive, robins2004robot, robins2009isolation}.  This is because the case study format is exploratory as opposed to evaluative in nature, employs both quantitative and qualitative data analyses, and deals with a small number of subjects \cite{yin2013case}.  For researches in human robot interactions for children with ASD, these qualities of case study are advantageous.  Firstly, HRI researches in children with ASD is a relatively novel field, not having a sound theoretical foundation, and which hypotheses are important to test are yet to be systematically established.  Conducting pilot studies that focus on generating important hypotheses in an exploratory manner rather than conducting those that focus on testing hypotheses not soundly founded are more effective in filling research gaps.  Secondly, since a major aspects of HRI research (similar to HCI research) deals with understanding the whys and hows of user decision making, qualitative data including researcher observations, interviews, focus groups, surveys, etc. are of as much importance as quantitative metrics.  Lastly, since each child with ASD is unique and vastly different from the rest of the spectrum, starting with pilot studies that focus on only a few cases / subjects in depth and iteratively move up the sample size to eventually aim to generalize across a population is preferential.

The case study format applies well to this thesis.  Our WoZ pilot study falls under HRI research for children with ASD, and our aim of generating important hypotheses and robot design recommendations in an exploratory manner makes the case study format highly suitable.  This does mean that compared to typical quantitative pilot studies (e.g. Bimbrahw et al. \cite{bimbrahw2012investigating}), we will be restricting to a smaller sample size, iteratively generating and testing hypotheses during data collection, and report the results in a more descriptive manner.
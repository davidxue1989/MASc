\section{Video Data Measures}
\label{sec:measures}
After annotating the video data, to investigate the impacts of the robot prompts on child's task completion and to measure the child's compliance to the prompts, the following metrics were counted:

\paragraph{Task Completion}
\begin{itemize}
	\item \textbf{Number of Steps Completed}: the number of hand-washing steps that were completed in a trial, the maximum being seven in each trial (these steps being turn on water, wet hands, get soap, scrub hands, rinse hands, turn off water, dry hands).
\end{itemize}

\paragraph{Child's Responses to Prompts}
\begin{itemize}
	\item \textbf{Complied Prompt Rate}: the ratio in percentage of prompts in a trial that the child complied versus prompted (complying, as judged by the annotator, is the child putting an effort of attempt following a prompt).
	\item \textbf{Ignored Prompt Rate}: the ratio in percentage of prompts in a trial that the child ignored versus prompted (ignoring, as judged by the annotator, is the child remaining unchanged in incorrect behavior following a prompt).
\end{itemize}

%\paragraph{Engagement and Visual Attention}
%\begin{itemize}
%	\item \textbf{Total Number of Times Child Smiles}: the number of prompt sections that the child smiled.
%	\item \textbf{Total Number of Times Child Murmurs}: the number of prompt sections that the child murmured.
%	\item \textbf{Looking at Prompting Agent Rate}: the percentage of prompt sections that the child looked at parent when parent was prompting or at robot when robot was prompting.
%\end{itemize}

\paragraph{} %a blank space
The specific definitions for each measure will be discussed further in Section \ref{sec:QuantitativeData_results} -- Quantitative Analysis and Results - Preliminary.  The difference between the terms ``prompts'' and ``prompt sections'' was discussed in the previous section \ref{sec:AnnotationFramework} -- Annotation Framework.
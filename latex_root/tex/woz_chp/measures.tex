\section{Video Data Measures}
\label{sec:measures}
After annotating the video data, to investigate the impacts of the robot prompts on child's step completion, to measure the child's compliance to the prompts, and to observe their relationships with the child's engagement level during hand-washing, the following metrics are counted:

\paragraph{Prompt Effectiveness}
\begin{itemize}
	\item \textbf{Total Number of Complete Steps}: the number of hand-washing steps that were completed.
	\item \textbf{Total Number of Parent Prompts}: the number of prompts delivered by the parent (e.g. verbal, pointing, motion demonstrations, nudging, guiding, and physically intervening).
\end{itemize}

\paragraph{Responses to Prompts}
\begin{itemize}
	\item \textbf{Compliance Rate}: the percentage of prompts that the child followed correctly.
	\item \textbf{Not Affected By Prompt Rate}: the percentage of prompts that the child ignored.
\end{itemize}

\paragraph{Engagement and Visual Attention}
\begin{itemize}
	\item \textbf{Total Number of Times Child Smiles}: the number of prompt sections that the child smiled.
	\item \textbf{Total Number of Times Child Murmurs}: the number of prompt sections that the child murmured.
	\item \textbf{Looking at Prompting Agent Rate}: the percentage of prompt sections that the child looked at parent when parent was prompting or at robot when robot was prompting.
\end{itemize}

\paragraph{} %a blank space
The specific definitions for each measure will be discussed further in the analysis section \ref{sec:VideoDataAnalysisAndResults}.  The difference between prompts and prompt sections was discussed in previous section \ref{sec:AnnotationFramework}.
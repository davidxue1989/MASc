\section{ASD Interventions}

Currently, interventions are mainly aimed to help children with ASD adjust more effectively to their environment \cite{francis2005autism}.  There are many approaches to ASD interventions, such as behavior based, pharmaceutically based, or dietary based \cite{francis2005autism}.  Also, there are interventions that focus on specific problems a child may experience, such as augmentative communication therapy, social skills teaching, sensory integration therapy \cite{francis2005autism}.  One reason for such a diverse collection of therapies to exist is because of the varying nature of ASD disabilities across individuals, making one single method not effective for the whole population, and sometimes not sufficient for an individual.  Consequently, it remains a challenge to show effectiveness of a method in a clinically significant sense, even if it is truly effective to a small sample of the population.  The most clinically tested, commonly practiced, and recommended methods are those based on the principles of Applied Behaviour Analysis \cite{foxx2008applied}.


\subsection{Applied Behaviour Analysis and Early Intensive Behavior Intervention}
Applied Behaviour Analysis (ABA) is based on scientifically proven theories of human behaviors, and is widely used for treating inappropriate behaviors and teaching new behaviors.  Its strategies for treating inappropriate behaviors include: finding and changing antecedents to inappropriate behaviors, ignoring the behavior, or negative reinforcement through punishment \cite{foxx1982decreasing}.  Its strategies for teaching new behaviors include: giving stimuli to child to elicit a new behavior, positive reinforcement (e.g. praise), and maintenance and generalization strategies to make sure the newly learned behavior is retained across time and settings \cite{foxx1982decreasing}.  For cuing stimuli, high level of consistency is needed.  For positive reinforcement, individually selected and strategically used motivators (e.g. praise, a hug, a check mark, a favorite activity) should be given immediately after appropriate behaviors.  For maintenance and generalization strategies, prompt fading and testing across context and settings can be used.


For some children with ASD, research has shown that early ABA based intervention with intensive and extensive sessions of minimum 30 hours a week for 2 years (also known as Early Intensive Behavioral Intervention (EIBI)) are proven to be successful for improving ASD outcomes \cite{howlin2009systematic}.  Meta analyses have been conducted to evaluate the efficacy of EIBI for children with ASD.  Three meta-analyses \cite{eldevik2009meta, reichow2009comprehensive, peters2011meta} found that, for children with ASD participating in EIBI versus those receiving other treatments or treatment as usual, an average medium to large effect size can be observed for IQ change, which may be considered clinically significant \cite{hojat2004visitor}.  Both Eldevik et al. \cite{eldevik2009meta} and Peters-Scheffer et al. \cite{peters2011meta} also found that a smaller effect size for adaptive behavior change can be observed.  However, there are large individual differences in treatment response for children with ASD and most children continue to require specialized services \cite{thill2012robot}.  All in all though, EIBI is the standard of care for children with ASD in North America \cite{keenan2014autism}.

\subsection{Daily Living Skill Teaching and the Discrete Trial Training}
In many children and adults with ASD, daily living skills are one of the key skills that still need to be learned through training and interventions.  In a study for adults with high functioning ASD by \cite{farley2009twenty}, they found that, among a range of variables including IQ, adaptive behavior measures (Vineland Adaptive Behavior Scales -– VABS \cite{sparrow2005vineland}) were the variables most closely and positively related to better outcome. Across the adaptive domains the ‘daily living skills’ domain was most highly associated with better outcome of quality of life.

There are three major cognitive theories developed that attempt to explain the underlying mechanics of the difficulties experienced by individuals with ASD: Theory of Mind deficit, executive dysfunctioning, and weak central coherence \cite{rajendran2007cognitive}.  The Theory of Mind deficit explains some individuals with ASD's inability to create an image of an agent's mental states (including one's own), thus the inability to infer what others think, believe, and desire, consequently to predict their behaviors.  Executive dysfunctioning explains some individuals with ASD's impairments in several higher-level capacities necessary for planning and monitoring of behaviors, set shifting, inhibiting automatic actions, and holding information in working memory.  Weak central coherence explains some individuals with ASD's cognitive style of local or detail-focused processing, leading to missing more global processing of information in context and for meaning.  These cognitive impairments in individuals with ASD are the underlying factors that cause some individuals to have difficulties learning and executing activities of daily living.  One approach to address this is through cognitive function trainings that focus on learning isolated cognitive skills that the individual lacks.  It has been found, however, that improvements in adaptive behaviors may not automatically result in improvements in cognitive skills \cite{chin2000teaching, teunisse2007cognitieve}.  Thus, instead of targeting the cognitive skills, which might not generalize to adaptive behaviors, the more direct approach would be to target the behaviors themselves, through interventions such as those based on the ABA framework.

Discrete trial teaching (DTT) \cite{howard2005comparison}, incidental teaching (IT) \cite{mcgee1986extension}, and pivotal response training (PRT) \cite{koegel2003teaching} are interventions effectively used in adaptive skill building in children with ASD and are based on ABA methodology \cite{palmen2013behavioral}.  DTT focuses on specific targeted behaviors, and relies on intensive and repetitive trainings with appropriate and immediate reinforcements.  IT focuses on teaching the child through natural means, emphasizing on having the child initiate learning opportunities.  For example, the therapist may hide one of the shoes of the child before it is time to go outside, thus creating an opportunity for the child to request verbally the missing shoe.  PRT philosophy also emphasizes naturalistic teaching.  Instead of focusing on individual behaviors, it targets ``pivotal'' areas of the child's development, such as motivations, response to multiple cues, self-management, and initiation of social interactions.  By targeting these key pivotal areas for improvement, broad area improvements can be seen in sociability, communication, behavior, and academic skill building.

Reviewing these methods, DTT is the most appropriate for teaching skills for a single activity of daily living, such as teaching hand-washing.  Many studies have shown effectiveness of EIBI using DTT, as reviewed by Case-Smith et al. \cite{case2008evidence}: The original study of DTT, by Lovaas, compared two groups of 19 young children with ASD, one receiving 40 hours per week of intensive DTT and the other receiving 10 hours or less of similar training, and showed improved IQ for the former after 2 years of intervention \cite{lovaas1987behavioral}.  There are many studies that confirmed these findings.  For one, in a more recent study, Cohen et al. compared two groups of 21 children with ASD, one receiving EIBI for one year and moved to less intensive services after, the other group receiving community-based services only, and showed higher IQ, language comprehension, and adaptive behavior for the former group by the end of 3 years \cite{cohen2006early}.  In addition, the DTT method is reproducible by nonprofessionals, as well.  Sallows et al. showed that parents who are taught proper implementation of DTT produced similar results with children with ASD as those produced by trained therapists after 4 years \cite{sallows2005intensive}.

\subsubsection{Implementation Details of DTT}
In summary, there are five steps to DTT \cite{bogin2010steps}:
\begin{enumerate}
	\item Grabbing the child's attention
	\item Discriminative stimulus: must be simple, clear, and concise, and wording must be consistent
	\item Child's response: stop incorrect response by instructing again immediately or provide prompts
	\item Prompt to aid the child if no response or wrong: this has the advantage of helping him quickly move through trials and avoid boredom and frustration.  Always use least intrusive prompt that ensures correct response.  Fade prompts to promote maintenance of effect and prevent dependence on intervention.  Types of prompts include: verbal, gesture, modeling, visual, physical.
	\item Reinforcement: given immediately after correct responses, tailored individually, always pair with verbal praise.
\end{enumerate}

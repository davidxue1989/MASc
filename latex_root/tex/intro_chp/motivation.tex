\section{Motivation}

\subsection{Children with ASD and the need for ATCs}

Autism spectrum disorder (ASD) is a complex neurological and developmental condition that is characterized by impairments in social communication and the presence of repetitive behaviours and restricted interests \cite{american2013diagnostic}.  Recent research suggested that the prevalence of ASD among children aged 8 years (the peak prevalence age), when the broad spectrum is considered, is as high as 1:68 in the United States \cite{baoi2014prevalence}.  Although there is no federal government monitoring system currently present in Canada that provides accurate estimate of its prevalence of ASD, we know that ASD is the most common form of neurological disorder and severe developmental disability in children \cite{autism2014facts}.

Some children with ASD may have difficulty in learning self-management skills and require continuous assistance from a caregiver to carry-out activities of daily living (ADLs).  Assistive technologies for cognition (ATCs) aim to assist, instruct, train, or educate individuals with cognitive disabilities to overcome challenges in life, including executing activities of daily living (ADL).  In particular, autonomous or self-operated ATCs promote independence for children with ASD in need of such assistance.  Commercially available devices such as computers, video players, audio players, cellphones, etc. have all been exploited as platforms for ATCs.


\subsection{COACH and its Challenges}

The COACH (Cognitive Orthosis for Assisting with aCtivites in the Home) system, developed by Mihailidis et al., is an autonomous prompting system \cite{mihailidis2008coach}.  It uses computer vision and artificial intelligence to automatically detect user actions when executing ADLs, and prompts appropriately when a user needs assistance.  It was first developed for the dementia population, but a version appropriate to the ASD population was recently adapted and tested in a pilot study \cite{bimbrahw2012investigating}.


The system currently uses audio and video prompts using an LCD monitor screen as its primary prompting modalities.  In the pilot study, the hand-washing activity is used as an example to test the system's effectiveness because of the simplicity of its tasks as well as the washroom settings being easily controlled.  The hand-washing activity is broken into five steps, with verbal prompts being: ``turn on the water and wet your hands'', ``put soap on your hands'', ``scrub your hands'', ``rinse your hands'', ``turn off the water and dry your hands''.  At each step, if the child does not successfully execute the correct action within a time threshold, the system prompts.  The prompting consist of first displaying a still image attention grabber on screen that attempts to capture child's attention, then playing the pre-recorded audio and/or video prompts.  The video prompts are videos recorded from point-of-view (first person) perspective of a person executing the step.  If child successfully executes a step with or without prompting, the system congratulates by saying ``Good job''.  If the child does not get it right within a time threshold, the system prompts again.  This is repeated until a predetermined maximum number of times before calling over the caregiver to assist.  


The pilot study of COACH for ASD showed good acceptance by the children with ASD and their caregivers \cite{bimbrahw2012investigating}.  System effectiveness was shown in that 78\% of the hand-washing steps were completed without caregiver assistances by the children themselves, all of whom were unable to wash-hands on their own before.


The major area of improvement identified in the study was to increase the child's engagement during prompting and task execution.  Firstly, almost half of the system's prompts were ignored by the children.  Assuming that these children were able to understand the prompts, as seen by their correct actions to the not ignored prompts, their noncompliance is a reflection of their disinterest in the prompts.  Secondly, several occasions of child being distracted or disinterested to tasks were observed.


To tackle this problem, two potential approaches can be applied: 1) change the prompting modality to one that's inherently more engaging to the child (e.g. humanoid robot); 2) track the child's visual focus of attention in real-time, and issue attention grabbers when child is distracted.

\begin{appendices}
\chapter{SRS Survey Result}
Choices for answers: 1 = Not True, 2 = Sometimes True, 3 = Often True, 4 = Almost Always True.
\begin{longtable}{ | p{13cm} | l | }
	\hline
	\textbf{Survey Question}	&	\textbf{Parent's Answer}	\\	\hline	\hline		
	1. Seems much more fidgety in social situations than when alone.	&	2	\\	\hline
	2. Expressions on his or her face don't match what he or she is saying.	&	2	\\	\hline
	3. Seems self-confident when interacting with others.	&	1	\\	\hline
	4. When under stress, he or she shows rigid or inflexible patterns of behavior that seem odd.	&	2	\\	\hline
	5. Doesn't recognize when other s are trying to take avantage of him or her.	&	4	\\	\hline
	6. Would rather be alone than with others.	&	4	\\	\hline
	7. Is aware of what others are thinking or feeling.	&	2	\\	\hline
	8. Behaves in ways that seem strage or bizarre.	&	3	\\	\hline
	9. Clings to adults, seems too dependent on them.	&	4	\\	\hline
	10. Takes things too literally and doesn't get the real meaning of a conversation.	&	3	\\	\hline
	11. Has good self-confidence.	&	2	\\	\hline
	12. Is able to communicate his or her feelings to others.	&	1	\\	\hline
	13. Is awkward in turn-taking interactions with peers (e.g., doesn't seem to understand the give-and-take of conversations).	&	1	\\	\hline
	14. Is not well coordinated.	&	2	\\	\hline
	15. Is able to understand the meaning of other people's tone of voice and facial expressions.	&	3	\\	\hline
	16. Avoids eye contact or has unusual eye contact	&	2	\\	\hline
	17. Recognizes wen something is unfair.	&	1	\\	\hline
	18. Has difficulty making friends, evn when trying his or her best.	&	4	\\	\hline
	19. Gets frustrated trying to get ideas across in conversations.	&	3	\\	\hline
	20. Shows unusual sensory interests (e.g., mouthing or spinning objects) or strange ways of playing with toys.	&	2	\\	\hline
	21. Is able to imitate others' actions.	&	3	\\	\hline
	22. Plays appropriately with children his or her age.	&	1	\\	\hline
	23. Does not join group actviities unless told to do so.	&	3	\\	\hline
	24. Has more difficulty than other children with changes in his or her routine.	&	2	\\	\hline
	25. Doesn't seem to mind being out of step with or ``not on the same wavelength'' as others.	&	3	\\	\hline
	26. Offers comfort to others when they are sad.	&	3	\\	\hline
	27. Avoids starting social interactions with peers or adults.	&	2	\\	\hline
	28. Thinks or talks about the same thing over and over.	&	3	\\	\hline
	29. Is regarded by other children as odd or weird.	&	3	\\	\hline
	30. Becomes upset in a situation with lots of things going on.	&	2	\\	\hline
	31. Can't get his or her mind off something once her or she starts thinking about it.	&	2	\\	\hline
	32. Has good personal hygiene.	&	3	\\	\hline
	33. Is socially awkward, even when he or she is trying to be polite.	&	3	\\	\hline
	34. Avoids people who want to be emotionally close to him or her.	&	3	\\	\hline
	35. Has troube keeping up with the flow of a normal conversation.	&	3	\\	\hline
	36. Has difficulty relating to adults.	&	2	\\	\hline
	37. Has difficulty relating to peers.	&	3	\\	\hline
	38. Responds appropriately to mood changes in others (e.g., when a friend's or playmate's mood changes from happy to sad).	&	2	\\	\hline
	39. Has an unusually narrow range of interests.	&	2	\\	\hline
	40. Is imaginative, good at pretending (without losing touch with reality).	&	1	\\	\hline
	41. Wanders aimlessly from one activity to another.	&	3	\\	\hline
	42. Seems overly sensitive to sounds, textures, or smells.	&	3	\\	\hline
	43. Separates easily from caregivers.	&	2	\\	\hline
	44. Doesn't understand how events relate to one another (cause and effect) the way other children his or her age do.	&	3	\\	\hline
	45. Focuses his or her attention to where others are looking or listening.	&	2	\\	\hline
	46. Has overly serious facial expressions.	&	2	\\	\hline
	47. Is too silly or laughs inappropriately.	&	2	\\	\hline
	48. Has a sense of humor, understands jokes.	&	2	\\	\hline
	49. Does extremely well at a few tasks, but does not do as well at most other tasks.	&	3	\\	\hline
	50. Has repetitive, odd behaviors such as hand flapping or rocking.	&	4	\\	\hline
	51. Has difficulty answering questions directly and ends up talking around the subject.	&	4	\\	\hline
	52. Knows when he or she is talking too loud or making too much noise.	&	1	\\	\hline
	53. Talks to people with an unusual tone of voice (e.g., talks like a robot ro like he or she is giving a lecture).	&	1	\\	\hline
	54. Seems to react to people as if they are objects.	&	2	\\	\hline
	55. Knows when he or she is too close to someone or is invading someone's space.	&	1	\\	\hline
	56. Walks in between two people who are talking.	&	3	\\	\hline
	57. Gets teased a lot.	&	2	\\	\hline
	58. Concentrates too much on parts of things rather than seeing the whole picture.  For example, if asked to describe what happened in a story, he or she may talk only about the kind of clothes the characters were wearing.	&	1	\\	\hline
	59. Is overly suspicious.	&	1	\\	\hline
	60. Is emotionally distant, doesn't show his or her feelings.	&	2	\\	\hline
	61. Is inflexible, has a hard time changing his or her mind.	&	2	\\	\hline
	62. Gives unusual or illogical reasons for doing things.	&	2	\\	\hline
	63. Touches others in an unusual way (e.g., he or she may touch someone just to make contact and then walk away without saying anything).	&	4	\\	\hline
	64. Is too tense in social settings.	&	3	\\	\hline
	65. Stares or gazes off into space.	&	4	\\	\hline
\caption{The SRS Survey Data}
\label{tab:SRSSurveyData}
\end{longtable}

\chapter{Post-Intervention Survey Result}
Choices for answers: Strongly Agree, Agree, Neither Agree nor Disagree, Disagree, Strongly Disagree.
\begin{table}[H]
	\centering
	\begin{tabular}{ | p{12cm} | l | }
		\hline
		\textbf{Survey Question}	&	\textbf{Parent's Answer}	\\	\hline	\hline		
		Hand-washing steps break down was appropriate	&	Strongly Agree	\\	\hline
		My child understood the verbal prompts	&	Agree	\\	\hline
		Robot's verbal prompts were appropriate	&	Strongly Agree	\\	\hline
		The prompt wordings were similar to mine	&	Strongly Agree	\\	\hline
		The prompt voice and tone were appropriate	&	Strongly Agree	\\	\hline
		The prompt wordings were easy to understand	&	Strongly Agree \\	\hline
		My child understood the gesture prompts	&	Strongly Agree	\\	\hline
		The gesture prompts were appropriate	&	Strongly Agree	\\	\hline
		The gesture prompts were easy to understand	&	Agree	\\	\hline
		The physical appearance of robot is aesthetically pleasing	&	Agree	\\	\hline
		The attention grabber gestures were appropriate	&	Strongly Agree	\\	\hline
		The verbal rewards were appropriate	&	Strongly Agree	\\	\hline
		The reward gestures were appropriate	&	Strongly Agree	\\	\hline
		The robot was effective in assisting my child through hand-washing	&	Strongly Agree	\\	\hline
		The robot motivated my child to wash hands	&	Strongly Agree	\\	\hline
		The robot was fun for my child to use	&	Strongly Agree	\\	\hline
		My child was confused by the robot	&	Disagree	\\	\hline
		I like the idea of a robot prompting my child	&	Strongly Agree	\\	\hline
		The robot is able to provide guidance as well as I can or better	&	Neither Agree or Disagree	\\	\hline
		I would want to own a robot like this one	&	Strongly Agree	\\	\hline
	\end{tabular}
	\caption{The Post-Intervention Survey Data}
	\label{tab:PostInterventionSurveyData}
\end{table}

\end{appendices}
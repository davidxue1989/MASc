\chapter{Conclusion}
In this thesis, we aimed to investigate a new prompting agent, humanoid robot NAO, for COACH during the prompting of hand-washing steps to a child with ASD.  A Wizard of Oz (WoZ) study was conducted, and yielded promising results.
%%.  In addition, we further improved COACH by attempting to implement a Visual Focus of Attention (VFOA) tracker.  
The thesis answers the hypotheses raised in the following way (the hypotheses are shown in bold):

\paragraph{The humanoid robot, NAO, is able to independently assist child with ASD through hand-washing, and child exhibits greater engagement level, higher prompt compliance, and better task completion when prompted by NAO than by parent.}
Through the WoZ study, we have seen that NAO was effective in facilitating task completion in both variety and quality of steps, approaching the level of effectiveness achieved by the parent, but not better.  Also, NAO had a low Complied Prompt Rate compared to the parent during the first phase when it was first introduced, but through a training phase of joint prompting with the parent, NAO resulted in a higher Complied Prompt Rate than before, comparable to that of the parent's.  We attribute the improvements of the robot effectiveness to the training phase, though several confounding variables such as learning effects, fatigue, and robot control were discussed.  In whole, although we have not achieved totally independent assistance using NAO, we have shown that NAO has very good potential of achieving independent assistance, given a longer and more intense training phase.

\paragraph{Gestural, gaze, and verbal are the essential modes of interactions present in the hand-washing prompting scenario between child with ASD and the prompting agent NAO.}
The WoZ study revealed that verbal instructions and pointing gestures are essential for a prompting agent to assist our participant, who knows the execution of each hand-washing step, but needs reminders of which step to execute.  In cases the child did need demonstrations for a step, NAO's limited dexterity as well as the slower motion speed of the motion demonstrations made it less effective compared to that of the parent's.  In addition, when the child is not complying, the parent increased the severity of the prompts, which the robot should implement in the future.  In terms of gaze behaviors during interactions, our participant generally avoided looking at the prompting agent when he knew what to do, and tends to look at the parent more than NAO when seeking help.  The child's gaze behavior was a poor indication of engagement.  Lastly, detection and understanding of verbal feedbacks from the child may be useful in predicting the engagement of our participant.

%%\paragraph{Using 3DMM and ALR for estimating head pose and eye pose, and using the Kinect camera, a classification rate of more than 80\% is achieved for estimating child's VFOA on NAO, monitor screen, soap, towel, tap region, hands, and idling.}
%%We have successfully completed a head pose tracker using the Kinect camera by modifying the KinFu algorithm.  However, due to rapid head movements of the participant during hand-washing sessions, our head pose tracker was not successful in tracking his head.  Future improvements were suggested.  We have also implemented frontal pose transformation of the head image for eye region cropping.  But due to limitation of time, we did not implement the eye pose tracker using EYEDIAP dataset employing the ALR method.  Lastly, we implemented the object under gaze estimator.  In whole, we showed the feasibility of implementing the VFOA's three components (i.e. head pose tracker, eye pose tracker, object under gaze estimator), and described the steps for completing its implementation and evaluating its performance.

\section{Significance}
In conclusion, the thesis results suggest promising prospects of utilizing the humanoid robot, NAO, as a novel prompting agent to enhance COACH, and to fill the gap of in-vivo ATCs for teaching daily skills to children with ASD following the ABA framework.  Future investigations in regards to improving the child's engagement further through more creative design of the robot's appearance and behavior dynamics were suggested and explored.
%%One such improvement could come from implementing robot behaviors contingent to the VFOA of the child.

\section{Thesis Process Reflections}
Looking back, there were several things the researcher could have done differently to have a more successful thesis.

The biggest difficulty the researcher experienced was the lacking of intuitions in regards to children with ASD behaviors and how to design robot behaviors accordingly.  This difficulty stems from the researcher's lack of experience with children with ASD, and not having backgrounds in psychology, cognitive science, or behavior science.  This kind of problem is common in engineering design when the engineering team lacks expertise in the field application of the system.  To compensate, common strategies include iterative design process that includes experts and end users early and test the system prototypes frequently and rapidly so we can learn and succeed efficiently.  In the case of this thesis, the researcher had the resource of the committee, composed of expert researchers in the field of ATC for autism.  The researcher should have utilized this resource more, by seeking their knowledge and intuitions early in the design process.  Similarly, a more formal approach to this is setting up focus groups with parents, caregivers, therapists, and clinicians to review design prototypes early in the design process, which the researcher did not do, either.  Lastly, the researcher could have involved pilot testing the system with typically developing children first.  The researcher only tested with adult volunteers, which posed a bias in the test because the adult volunteers were all enthusiastic, lacking the opportunity to test scenarios of noncompliance from the user, especially due to lack of motivation and engagement.

The researcher was not trained in qualitative research, and originally hoped to bypass the explorative case study (qualitative) stage, and directly conduct quantitative studies evaluating system efficacy.  Reflecting back, attempting to jump steps was progress hindering and led to inefficiency in learning what works in our design.  The explorative case study stage was a necessity because of the need to validate our assumptions and the need to gain intuitions on what measures and hypotheses are appropriate and meaningful for evaluative quantitative studies.  Furthermore, the research field of HRI for children with ASD is in its infancy.  It lacks theoretical frameworks that guide design decisions based on user behavior results.  In addition, the literatures in this field up to date offered very few descriptions on specific robot behaviors that are appropriate for facilitating both engagement and compliance from children with ASD.  All in all, hoping to produce successful results by skipping directly to a evaluative quantitative study with little grounding on our design decisions was a long shot, especially our use of humanoid robot to prompt children with ASD for daily living activity was the first research of its kind.  The researcher should have realized this from the start, and time could have been better allocated, investing less in learning quantitative research methods, not analyzing data for quantitative measures that yielded little significance, and better preparation for the case study and the qualitative research method.

Lastly for the clinical contribution, a larger portion of time should have been allocated for conducting the case study.  Our study results suggest the possibility that our participant would have demonstrated further improvement in robot compliance if the training phase (Phase C) was longer, and we would have greater confidence on the maintenance of such improvement if the robot testing phases (Phase B1 and B2) were longer, too.  In addition, the researcher wished the recruitment of subjects happened earlier, so that more than one subjects could have been recruited.  Our participant knew most of the hand-washing steps, thus although we showed improvement of compliance on our participant using the current design, the same design recommendations might not generalize to the population.  For a participant not knowing any hand-washing steps, the focus of robot prompts need to be shifted from step initiation to step motion teaching.  Finally, having a pre-test session where the child was observed to wash hands by himself with neither helps from the parent nor the robot would have added valuable grounding to our results.

%%For the technical contribution section of the thesis, instead of implementing three modules of the VFOA tracker (a head pose tracker, an eye pose tracker, and an object identification algorithm), the researcher should hvae decreased the scope to head tracking module only.  This way, instead of running out of time investigating possible ways to implement the other two modules, the head tracking module could have been thoroughly investigated, including improving the eye cropping submodule to real-time and conducting proper evaluation of our module implementation's accuracy and robustness.

In all, the researcher learned a lot of valuable lessons, both in regards to designing robot behaviors for children with ASD, and in regards to conducting qualitative and quantitative research.  The researcher hopes this thesis may be beneficial for anyone who wishes to continue pursuing building more engaging and effective ATCs for children with ASD.
%%Although many mistakes were made, the researcher is reminded of the Chinese phrase: ``Failure is the mother of success'', and hopes this thesis has been a good mother.
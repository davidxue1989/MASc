\section{Research Design and Strategy}
One major objective of this thesis is to investigate the impacts that using a humanoid prompting agent has on engagement, prompt compliance and task performance of children with ASD during hand-washing activities.  This is the first research of its kind in the field of humanoid robot prompting agent guiding children with ASD through an activity of daily living.  Therefore, it is wise to begin with a pilot study, the purpose of which is to show plausibility of the key underlying assumptions of our hypotheses, to create design recommendations of the robot for future studies, and to probe what questions are important to be answered later in a more rigorous randomized control trial.  For these reasons, the pilot study should be exploratory in nature, having a flexible experimental design, and a relatively low experiment setup cost.  A Wizard of Oz experimental design in a case study format is appropriate.

\subsection{Wizard of Oz Experimental Design}
The Wizard of Oz (WoZ) is an experimental design widely used in Human Computer Interaction (HCI) and Human Robot Interaction (HRI) research.  In a typical WoZ study, there is an interactive agent that is not yet fully autonomous, and is remotely controlled by a human operator (i.e. the ``wizard''), and this fact is concealed from the user being tested until after the study.  The wizard may control one or many parts of the agent, such as speech recognition and understanding, affect recognition, dialog management, utterance and gesture generation and so on \cite{bhargava2013demonstration}.  The advantage of a WoZ study is that it does not require a large amount of work spent in implementing the artificial intelligence (AI) behind the agent before testing it in a study -- it is mocked up by the wizard remote controlling the agent.  This is great for testing hypotheses early on in the design loop, enabling us to obtain feedbacks from users, learn, and iterate through design cycles faster.  Of course, care needs to be taken to ensure the mocked up part of the AI is implementable in the near future, since the real purpose of the mock up is to have an early knowledge of the real design constraints, not trying to provide a less constrained solution.

The characteristics of a WoZ study fits our pilot study requirements, where we want to learn early the important design questions regarding building an effective ADL prompting robotic agent for the children with ASD population.  Thus, we have conducted a WoZ study in which a humanoid robot whose motions and speech were preprogrammed, but the decision and timing of their executions were controlled remotely by the researcher.  The researcher, acting as the wizard, is mocking up the computer vision algorithms that understand the child with ASD's actions, the speech recognition algorithms that recognize the child's verbal interactions, and the AI decision making algorithms that decide what prompts to deliver and when to deliver them.

During each WoZ study trial, the child with ASD was asked to complete the hand-washing activity in the washroom with the supervision of one of his/her parents, with the help of the NAO robot, or with the help of both the parent and the robot. The researcher and the parent were in an adjacent room out of the child's view to observe his/her hand-washing activity.  However, the parent could enter the washroom if the child needed physical assistance to complete a step. A controlling interface running on a laptop, connected wireless to the robot, was used by the researcher to remotely control the robot, as well as to monitor the progress and responses of the child through the video feeds of the cameras installed in the washroom.

\subsection{Case Study Format}
The WoZ study was conducted, analyzed, and reported in a case study format.  A case study is defined by Yin in a two fold definition: ``a case study is an empirical inquiry that investigates a contemporary phenomenon in depth and within its real-life context'' and its result ``relies on multiple sources of evidence, with data needing to converge in a triangulating fashion'' \cite{yin2013case}.  The case study format is widely used in human robot interaction research for children with ASD \cite{kozima2005interactive, robins2004robot, robins2009isolation}.  This is because the case study format can be exploratory as opposed to evaluative in nature, employing both quantitative and qualitative data analyses, and dealing with a small number of subjects \cite{yin2013case}.  For research in human robot interactions for children with ASD, these qualities of a case study are advantageous.  Firstly, HRI research in children with ASD is a relatively novel field, not having a sound theoretical foundation, and it is yet to be systematically established which hypotheses are important to test.  Thus, it is more effective in filling research gaps by conducting pilot studies that focus on generating important hypotheses in an exploratory manner rather than conducting those that focus on testing hypotheses not soundly founded.  Secondly, since a major aspects of HRI research (similar to HCI research) addresses understanding the whys and hows of user decision making, qualitative data including researcher observations, interviews, focus groups, surveys, etc. are of as much importance as quantitative metrics, if not more, during the early stages of research.  Lastly, since each child with ASD is unique and vastly different from the rest of the spectrum, starting with pilot studies that focus on only a few cases / subjects in depth and iteratively move up the sample size to eventually aim to generalize across a population is preferential.

The case study format applies well to this thesis.  Our WoZ pilot study falls under HRI research for children with ASD, and our aim of generating important hypotheses and robot design recommendations in an exploratory manner makes the case study format highly suitable.  This does mean that comparing to typical quantitative pilot studies (e.g. Bimbrahw et al. \cite{bimbrahw2012investigating}), we needed to restrict to a smaller sample size, to iteratively generate and test hypotheses during data collection, and to report the results in a more descriptive manner.

\subsection{Combining Qualitative and Quantitative Analyses}
Qualitative and quantitative analyses in this thesis' case study play complimentary roles with each other.  From a result description point of view, there are certain aspects of the study best summarized by stating qualitatively the themes or trends observed, while other aspects of the study demand accounting of metrics whose trends (or lack of trends) are presented by graphical plots.  The decisions of which results are presented qualitatively and which quantitatively will be discussed more in Section \ref{Sec:DataAnalysisAndResults} -- Data Analysis and Results.  From a hypotheses generation and validation point of view, qualitative analyses provide the intuitive understanding and interpretation of our study results, while quantitative analyses provide further investigation and validation to the hypotheses generated by the qualitative analyses.  Through observations and discussions of the child with ASD's behaviors in response to robot and parent prompts, qualitative analyses may identify important trends and themes that can be validated using quantitative analyses.  The choice of metrics used in quantitative analyses are, in turn, dependent on the qualitative observation results.  This is coherent to the triangulation approach that qualitative research employs, which assumes that the internal validity of a research is strengthened when the use of multiple sources of data, multiple methods, multiple investigators, or multiple theories confirms the same findings \cite{merriam2014qualitative}.  Quantitative analyses, then, can be the ``one more method'' that confirms and strengthens the validity of the findings from the qualitative research.

One challenge of conducting qualitative and quantitative analyses alongside in this complimentary manner within a study is that, for qualitative research, data analysis is conducted during data collection, and data collection method is revised iteratively with ongoing analysis \cite{merriam2014qualitative}.  For example, if analysis reveals that a certain device should be improved or a certain data collection protocol should be changed, data collection may take a turn in this new direction.  This may potentially create problems for quantitative analysis if it shares with qualitative analysis the same pool of data collected.  For the data collected using methods that are changing, grouping them would be not rigorous if the changes may significantly affect our outcome, but not grouping them would make the sample size very small for each group.  For this thesis, however, this may be a compromise we are willing to accept for the pilot study, since the validity of our results rest not solely on quantitative analyses.  It is planned for future studies to conduct a statistically significant quantitative research for rigorously validating the hypotheses and design recommendations generated by this thesis.

\subsection{Case Selection and Sample Selection}
\label{sec:CaseSelectionAndSampleSelection}
When conducting a case study, selecting the right case and sampling in it the data to focus on are essential for bounding the scope of the study and producing results of significance.  The case selection and the sampling within the case are essentially two layers of the same process called sample selection \cite{merriam2014qualitative}.  Merriam \cite{merriam2014qualitative} listed several sample selection methods.  Of them, the ones most relevant to a case study are typical sampling, unique sampling, maximum variation sampling, and theoretical sampling. As Merriam defines them:
\begin{itemize}
	\item \textbf{Typical Sampling} chooses samples that are the ``average person, situation, or instance of the phenomenon of interest''
	\item \textbf{Unique Sampling} focuses on ``unique, atypical, perhaps rare attributes or occurrences of the phenomenon of interest''
	\item \textbf{Maximum Variation Sampling} explores extreme variations of the phenomenon of interest, sometimes even involving ``a deliberate hunt for negative or disconfirming variations''
	\item \textbf{Theoretical Sampling}  is ``an ongoing sample selection process'' that ``begins the same way as purposeful sampling'' (purposeful sampling includes the three sampling methods above), ``but the total sample is not selected ahead of time'' but is decided iteratively as results from data analysis roll out
\end{itemize}

In our WoZ pilot case study's context, case selection was the process of setting inclusion criteria for recruiting a subpopulation of the children with ASD, and sample selection within the selected case was deciding the robot and parent prompting protocols that are used to guide our participant through the hand-washing trials.  Case selection is be discussed in more detail in Section \ref{sec:Recruitment} - Recruitment.  Sample selection is discussed in more detail in Section \ref{sec:StudyProtocol} - Study Protocol.



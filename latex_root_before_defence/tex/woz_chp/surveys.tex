\section{Surveys}
In addition to observing the child with ASD interacting with the robot, surveys were designed to probe the child's background, experiences, and preferences, putting the observations into context.  Also, surveys probing the parent's opinion of the robot and the COACH system were also created, since the parent is also an important decision maker in the robot and the COACH system's design.

\paragraph{Social Responsiveness Scale Survey and Entrance Survey}
The Social Responsiveness Scale Survey \cite{constantino2002social} serves to screen participants during recruitment and was filled out by the parent that accompanied the child to all trials.  The entrance survey was administered along with the SRS survey.  It reports the child's demographics as well as inclusion criteria fit.  In addition, it also reports child's previous experiences with technologies, child's personal preferences, child's abilities on hand-washing and on other ADLs, and parent's expectation and concerns.

\paragraph{Post-Intervention Survey}
During the last visit, the same parent who completed the SRS survey and the entrance survey was be asked to fill out the post-intervention survey.  The survey reports the parent's opinion on the robot prompting system, giving suggestions and comments and how effectiveness and appropriate the system is, and any improvements or adjustments needed.

\section{ASD Interventions}

Currently, there is no cure for ASD, with all interventions aimed to help children with ASD adjust more effectively to their environment \cite{francis2005autism}.  There are many approaches to ASD interventions, such as behavior based, pharmaceutically based, or dietary based \cite{francis2005autism}.  Also, there are interventions that focus on specific problems a child may experience, such as augmentative communication therapy, social skills teaching, sensory integration therapy \cite{francis2005autism}.  One reason for such a diverse collection of therapies to exist is because of the varying nature of ASD disabilities across individuals, making one single method not effective for the whole population, and sometimes not sufficient for an individual.  Consequently, it remains a challenge to show effectiveness of a method in a clinically significant sense, even if it is truly effective to a small sample of the population.  The most clinically tested, commonly practiced, and recommended methods are those based on the principles of Applied Behaviour Analysis \cite{foxx2008applied}.


\subsection{Applied Behaviour Analysis and Discrete Trial Training for Skill Teaching}
Applied Behaviour Analysis (ABA) is based on scientifically proven theories of human behaviors, and is widely used for treating inappropriate behaviors and teaching new behaviors to people with cognitive disabilities.  Its strategies for treating inappropriate behaviors include: finding and changing antecedents to inappropriate behaviors, ignoring the behavior, or negative reinforcement through punishment \cite{foxx1982decreasing}.  Its strategies for teaching new behaviors include: giving stimuli to child to elicit a new behavior, positive reinforcement (e.g. praise), and maintenance and generalization strategies to make sure the newly learned behavior is retained across time and settings \cite{foxx1982decreasing}.  For cueing stimuli, high level of consistency is needed.  For positive reinforcement, individually selected and strategically used motivators (e.g. praise, a hug, a check mark, a favorite activity) should be given immediately after appropriate behaviors.  For maintenance and generalization strategies, prompt fading and testing across context and settings can be used.


For children with ASD, research has shown that early ABA based intervention with persistent and intensive sessions of minimum 30 hours a week for 2 years (a.k.a. Early Intensive Behavioral Intervention (EIBI)) are proven to be successful for improving ASD outcomes \cite{howlin2009systematic}.

Discrete Trial Training (DTT) is an ABA based method widely used for teaching children with ASD new behaviors, and particularly useful in teaching skills of daily living \cite{smith2001discrete}.  Many studies have shown effectiveness of EIBI using DTT, as reviewed by Case-Smith et al. \cite{case2008evidence}: The original study of DTT, by Lovaas, compared two groups of 19 young children with ASD, one receiving 40 hours per week of intensive DTT and the other receiving 10 hours or less of similar training, and showed improved IQ for the former after 2 years of intervention \cite{lovaas1987behavioral}.  There are many studies that confirmed these findings.  For one, in a more recent study, Cohen et al. compared two groups of 21 children with ASD, one receiving EIBI for one year and moved to less intensive services after, the other group receiving community-based services only, and showed higher IQ, language comprehension, and adaptive behavior for the former group by the end of 3 years \cite{cohen2006early}.  In addition, Sallows et al. showed that parents who are taught proper implementation of DTT produced similar results with children with ASD as those produced by trained therapists after 4 years \cite{sallows2005intensive}.

\subsubsection{Implementation Details of DTT}
There are five steps to a DTT \cite{bogin2010steps}:
\begin{enumerate}
	\item Grabbing the child's attention
	\item Discriminative stimulus: must be simple, clear, and concise, and wording must be consistent
	\item Child's response: stop incorrect response by instructing again immediately or provide prompts
	\item Prompt to aid the child if no response or wrong: this has the advantage of helping him quickly move through trials and avoid boredom and frustration.  Always use least intrusive prompt that ensures correct response.  Fade prompts to promote maintenance of effect and prevent dependence on intervention.  Types of prompts include: verbal, gesture, modeling, visual, physical.
	\item Reinforcement: given immediately after correct responses, tailored individually, always pair with verbal praise.
\end{enumerate}

%AM: I would like to see this section developed a bit more, including looking at more details on why this approach is important to follow, some more examples of clinical studies that have proven its efficacy, and then how this may map to the COACH and the importance of doing so.

%why DTT is important to follow:
%	- examples of clinical studies that proves its efficacy
%	- how does it map to COACH
%	- why is it important to map it to COACH

\subsection{Discussion}
\label{sec:DTTDiscussion}
To promote consistency and effectiveness, the implementations of ATCs for assisting children with ASD with daily living tasks should also follow the ABA framework, and model after the DTT steps.  Apparently, an ATC acts as either a substitute or a complement to the caregiver in guiding the child with ASD through ADLs in the home / community settings.  The goals for a successful ATC for children with ASD are two fold: one, to guide the child through an ADL without dependence on the caregiver; and two, to train the child to perform the steps of the ADL with fading dependence on the ATC.  We see the significance of DTT in skill teaching for children with ASD.  Thus, an ATC incorporating DTT would potentially not only reduce caregiver burden, but would implement clinically proven therapeutic interventions in the home / community settings.  In fact, this is exactly what COACH aims to achieve in its prompting design.  The following are the steps of how COACH prompts:
\begin{enumerate}
	\item Attention Grabber
	\item Verbal Instruction
	\item Video Prompt
	\item Child's Response
	\item Verbal Reward
\end{enumerate}

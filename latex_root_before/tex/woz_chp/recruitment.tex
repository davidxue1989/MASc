\section{Recruitment}
\label{sec:Recruitment}
Since children with ASD are a diverse population with each child unique and vastly different from one another, the subpopulation from which we recruit from directly influences our study results and focus.  This is the process of case selection.  Since the aim of our study is to form important hypotheses about children with ASD that aid in future iterations of the robot prompting design, focusing on the typical use case would be most beneficial at the pilot study stage.  One way to ensure this is through sampling for the typical user that our robot is designed for.  This would be children with ASD, not adults nor any one without a diagnosis of ASD.  In addition, the participant needs to be able to interact with a prompting agent, understanding and following its instructions.  Gender, ethnicity, and other factors that are not significant in determining important robot design recommendations are not constrained in our inclusion criteria.  In addition, hand-washing skills are constrained to a certain degree, but the level of hand-washing skills need not be non-existent.  All participants yield valuable design input information as long as the participant requires some form of assistance during hand-washing, whether it be requiring prompts to get started, to know which steps to execute, or to know how to execute each step.  This means, on the spectrum of hand-washing levels, we might have several different cases depending on the participants we recruited.

Participants were recruited from a list of families of a previous autism study that indicated that they would be interested in participating in future studies related to the development of the COACH prompting system.

Participants were to be children between the ages of 4 to 15 with a diagnosis of ASD, and their parent. Three children would be recruited. This sample size is typical for studies of this nature (i.e. HRI case study for children with ASD), e.g. Kozima et al. had two participants in
\cite{kozima2005interactive}, Robins et al. had three in \cite{robins2004robot} and \cite{robins2009isolation}.  Participant demographics would be recorded and would include age, sex, and the Social Responsiveness Scale (SRS) test results.  The SRS is a commonly used tool to identify the presence and estimate the severity of ASD \cite{constantino2002social}. The results of the SRS would allow the research team to substantiate a diagnosis of an ASD for the child participants before proceeding with the study.

\paragraph{The \textbf{inclusion criteria} for enrolling in the study were as follows:}
\begin{itemize}
	\item Boys and girls between the ages of 4-15
	\item Parent report of a clinical diagnosis of an ASD – to be confirmed through administration of the Social Responsiveness Scale (SRS)
	\item Has difficulty independently completing self-care activities, specifically hand-washing
	\item Has the ability to follow simple, one-step verbal instructions
	\item Ethical consent  granted by parents or primary guardian
	\item Does not exhibit severely aggressive behavior
\end{itemize}

Each participating family were given a \$200 honorarium per child subject upon completion of the study. All participants were able to withdraw from the study at any time. The honorarium would then be adjusted to be proportionate to the number of visits completed (e.g. completing 3 visits means the participated child would receive \$100 (\$200 * 3 / 6 = \$100)). This would be made clear to participants at the time of consent.
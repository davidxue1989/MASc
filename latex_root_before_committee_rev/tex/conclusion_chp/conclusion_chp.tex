\chapter{Conclusion}
In this thesis, we aimed to investigate a new modality of interaction for COACH, using a humanoid robot NAO, during the prompting of hand-washing steps to a child with ASD.  A Wizard of Oz (WoZ) study was conducted, and yielded promising results.  In addition, we further improved COACH by implementing a Visual Focus of Attention (VFOA) tracker.  The thesis answers the hypotheses raised in the following way (the hypotheses are shown in bold):

\paragraph{The humanoid robot, NAO, is able to independently assist child with ASD through hand-washing, and child exhibits greater engagement level, higher prompt compliance rate, and better task completion when prompted by NAO than by parent.}
Through the WoZ study, we have seen that NAO was effective in facilitating task completion, though the parent was more effective.  Also, NAO had a low prompt compliance rate compared to the parent during the first phase it was introduced, but through a training phase of joint prompting with the parent, NAO resulted in a higher prompt compliance rate, comparable to that of the parent's.  The measures employed for engagement level (i.e. Number of Times Child Smiles, Number of Times Child Murmurs, Looking at Robot/Parent Rate) were very noisy, and the parent mostly had higher levels than the robot.  We did not observe any correlations between these measures to prompt compliance rate either.  In whole, although we have not achieved totally independent assistance using NAO, we have shown that NAO has very good potential of achieving independent assistance, given longer training and testing sessions.

\paragraph{Gestural, gaze, and verbal are the essential modes of interactions present in the hand-washing prompting scenario between child with ASD and the prompting agent NAO.}
The WoZ study revealed that verbal instructions and pointing gestures are useful for our participant, who knows the execution of each hand-washing step, but needs reminders of which step to execute.  In cases the child did need demonstrations for a step, NAO's limited dexterity made the motion demonstration somewhat confusing to the child compared to that of the parent's.  In terms of gaze behaviors during interactions, our participant generally avoided looking at the prompting agent when he knew what to do, and tends to look at the parent more than NAO when seeking help.  Lastly, when the child is not complying, the parent had the ability to increase the invasiveness of the prompt by physically influencing the child through nudging, guiding the arm, and fully doing the step hand-on-hand.  This increase of physical invasiveness was effective in promoting compliance, and NAO lacked such interaction modality.

\paragraph{Using 3DMM and ALR for estimating head pose and eye pose, and using the Kinect camera, a classification rate of more than 80\% is achieved for estimating child's VFOA on NAO, monitor screen, soap, towel, tap region, hands, and idling.}
We have implemented the head pose tracker using the Kinect camera, employing our modified version of the KinFu algorithm.  However, due to rapid head movements of the participant during hand-washing sessions, our head pose tracker was not successful in tracking his head.  Future improvements were suggested.  We have also implemented front-pose transformation of the head image for eye region cropping.  But due to limitation of time, we did not implement the eye pose tracker using EYEDIAP dataset employing the ALR method.  Lastly, we implemented the object under gaze estimator.  In whole, we showed the feasibility of implementing the VFOA's three components (i.e. head pose tracker, eye pose tracker, object under gaze estimator), and described the steps for completing its implementation and evaluating its performance.
\\
\\
In conclusion, the thesis results suggest promising prospects of utilizing the humanoid, NAO, as a novel prompting modality to enhance COACH, and to fill the gap of in-vivo ATCs for teaching daily skills that can automatically deliver prompts and reinforcement schedules.  Future investigations in regards to improving the child's engagement further through more creative design of the robot's appearance and behavior dynamics were suggested.  One such improvement could come from implementing robot behaviors contingent to the VFOA of the child.
\section{Roadmap}

This thesis will address how to improve the visual focus of attention of children with ASD when prompted by COACH.  It will approach the problem by incorporating a humanoid robot, NAO, and implementing algorithms to track and react to the child's gaze, in attempts to better engage children during ADL prompting sessions.


The thesis report will be broken into the following sections:


In Chapter 2, we first review the relevant literature on the established frameworks for ASD interventions, especially those that teach skills of daily living, and relate it to our prompting framework.  Next, we review research for ASD interventions that utilize technologies such as videos, computers, and robots, confirming this thesis' approach and identifying the research gap.  Also, our research direction is put into context in the Human Robot Interaction (HRI) field.  Lastly, we review state of the art research in visual focus of attention estimation, targeting key methodologies appropriate for this thesis's goal.


In Chapter 3, the research objectives and hypotheses are summarized.

In Chapter 4, the Wizard of Oz pilot study is presented.  The study setup, protocol, data collected, analysis and results are discussed in details.

In Chapter 5, the investigated algorithms for tracking a person's visual focus of attention are presented.  Specifically, the problem is broken into head pose estimation, eye pose estimation, and object identification.  The methods and evaluation for each are discussed.

In Chapter 6, we summarize and conclude the thesis.



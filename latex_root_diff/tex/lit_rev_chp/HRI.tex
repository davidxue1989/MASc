\section{Human Robot Interactions (HRI)}

HRI is a recent field of research focusing on natural and efficient interactions between humans and robots.  It is a multidisciplinary field with contributions from human computer interactions (HCI), artificial intelligence, robotics, design, social sciences, etc.  The study of HRI is important for designers of robot behaviors, especially if one pursuits a framework of user centered design, where user experience is given as much value as functionalities. 


\subsection{Socially Interactive Robotics (SIR)}
SIR is first defined by Fong et al. as robots for the main purpose of interacting with users without physical contact \cite{fong2003survey}.  The application domains where SIR are desirable are: robot as mediator of human-human interactions, robot as representations of humans, robot as companions to human, robot as modeling tool for researchers studying embodied social behaviors.  The kinds of social interactions SIR could simulate include: artificial emotions, speech, facial expression, body language and gesture.  We will refer to these as different modes of human-robot interactions through out this thesis.


\subsection{The Issue of Embodiment}
One major issue in the study of SIR (and that of HIR in general) is the issue of embodiment, i.e. what effects does the robot's physical presence have on its interactions with humans.  Ziemke \cite{ziemke2001disentangling} defined four levels of embodiment from least to most:
\begin{enumerate}
	\item \textbf{Structural coupling}: the presence (doesn't have to be physical) of the agent can influence a human's state
	\item \textbf{Physical embodiment}: the agent has a physical body
	\item \textbf{Organismoid embodiment}: the agent's physical body is organism-like
	\item \textbf{Organismic embodiment}: except the agent's organism-like body is of autopoietic, living systems
\end{enumerate}


The impact of embodiment in SIR is significant, as argued by Mataric \cite{mataric2005role}.  Because humans irrepressibly attribute human-like characteristics to embodied agents that are similar to them, the factors influencing the embodiment of the robot will impact significantly the engagement of the human in HRI.


Similar intuition is explored by Young et al \cite{young2011evaluating}.  As Young et al. pointed out, because the robot has an embodied presence in close proximity to the user, it has a greater influence than the disembodied counterpart on the user physically, emotionally, and socially. Thus, factors in embodiment has a large influence on how the user perceives the robot's identity and their relationship.  Young et al. provided a three level perspective on how to analyze these factors in embodiment in context of HRI:
\begin{enumerate}
	\item \textbf{Intrinsic level}: static factors regarding the appearance of the robot set the initial emotional response from the user.  For example, a cute looking robot animal may create instant affinity from children towards the robot.  Static qualities of the robot's voice and motion (e.g. amplitude, smoothness, speed, etc.) fall under this level, too.  In addition, factors in this level contribute to the role user perceives the robot to be in.  For example, a humanoid is better than an animal looking robot as a supervisor or mentor.
	
	\item \textbf{Behaviour level}: dynamic factors regarding the facial expressions, voice intonations, motion gestures and gaze, etc. modulate emotional responses from the user and interactions between user and robot.  In addition, these factors contribute to the robot's perceived role as well.
	
	\item \textbf{Role level}: the goal of the robot, and the planning and decision making of the robot's behaviors to achieve that goal (along with factors from previous two levels), determines the perceived role of the robot.  The user may treat the robot as a supervisor, mentor, companion, cooperative peer, slave, tool, etc. \cite{goodrich2007human}.
\end{enumerate}


\subsection{Socially Assistive Robotics (SAR)}
Feil-Seifer and Mataric defined SAR as the intersection of socially interactive robotics (SIR) and assistive robotics (AR) \cite{feil2005defining}.  In the past, AR has been mainly been used to describe robots that assisted people with physical disabilities through physical contacts.  But SAR arises from the need for robots to assist people with cognitive disabilities without using physical contacts.  In this context, SAR assumes an expert (mentor / coach / supervisor / teacher / assistance) role, and prompts the user through executions of tasks.  Feil-Seifer and Mataric pointed out that SAR has two (possibly conflicting) goals –- engaging the user to HRI and engaging the user to executing tasks.  SAR needs careful designing to satisfy both the social interaction goal and the assistive goal.  However, on a deeper level, the SIR goal is serving the AR goal in that, better social interactions would cause better user engagement during the co-operative activity, and yield better user compliance to prompts and ultimately better task performance.


Wainer et al. conducted studies along this line of hypothesis, evaluating user engagement through self-reflective survey and evaluating task performance through optimality of move during the task of solving a Towers of Hanoi puzzle \cite{wainer2007embodiment}.  They tested the difference between embodied robot versus its remote presence and its simulated virtual avatar (both disembodied versions are displayed on a computer screen).  They have found that people prefer interacting with the embodied robot and reported it being more helpful and watchful than the other two disembodied counter parts.  However, there were no significant improvements in task performances from using the embodied robot over the disembodied versions.


\subsection{Discussion}
This lack of impact on task performance in Wainer et al.'s study may be due to the learning effect when solving the puzzle and the novelty factor of using the robot.  Also, the use of such SAR on the clinical population and on clinically relevant tasks may yield a better result because of greater need for user engagement and user compliance in that context.  For our thesis, we will investigate the impact of an embodied robot on the clinical population of children with ASD in the task of hand-washing, where user's lack of compliance to prompts and of engagement in task executions are observed problems.  One thing to note is, we cannot conduct self-reflective surveys due to the nature of the children with ASD population.  Therefore, we will rely on objective measures for user engagement such as verbal responses and physical behaviors (proximity, posture, eye gaze and fixation) \cite{mataric2005role}.


